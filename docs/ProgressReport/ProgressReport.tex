\documentclass[12pt]{report}
\usepackage[utf8]{inputenc}
\usepackage{amsmath}
\usepackage{amsfonts}
\usepackage{amssymb}
\author{Edward Seabrook} 
\title{Third Year Project Progress Report}

\usepackage{titlesec}
\titleformat{\chapter}% reformat chapter headings
    [hang]% like section, with number on same line
    {\Large\bfseries}% formatting applied to whole
    {\thechapter}% Chapter number
    {0.5em}% space between # and title
    {}% formatting applied just to title

 %Get rid of margins?
 \usepackage[top=2.4cm, bottom=2.4cm, left=3.5cm, right=2.4cm]{geometry} 

\usepackage{todonotes}
\setlength{\marginparwidth}{3cm}
\reversemarginpar

\usepackage{hyperref}

\usepackage{fancyhdr} 
\pagestyle{fancy} 
\renewcommand{\headrulewidth}{0pt} 
\lhead{}\chead{}\rhead{}
\lfoot{}\cfoot{\thepage}\rfoot{}

\begin{document}


\begin{titlepage}

\begin{center}


% Upper part of the page
%\includegraphics[width=0.15\textwidth]{./logo}\\[1cm]    

\LARGE Electronics and Computer Science\\
Faculty of Physical and Applied Sciences\\
University of Southampton
\\[1.5cm]

\href{mailto:ejfs1g10@ecs.soton.ac.uk}{Edward JF Seabrook}\\[0.5cm]

\today \\[1cm]
{\bfseries A Tool to Simplify Network Administration in the Modern Home}\\[1.5cm]

\vfill

% Author and supervisor
\large
Project Supervisor: 
Dr.~T \textsc{Chown}\\

\large
Secondary Examiner:
Dr.~KP \textsc{Zauner} 

\vfill

A Project Progress Report Submitted for the Award of Computer Science

\end{center}

\end{titlepage}



\begin{abstract}
\em No more than 200 words. \em


\end{abstract}

\tableofcontents
\clearpage
\listoftodos

\chapter{Project Description}
I left my initial project definition very open ended as I was undecided between
two projects - I was unsure of the scope of a third year project, and my
competence as a software enineer. The original project definition was to create
``a tool to simplify network administration in the home''. The first potential
project was a home gateway monitoring system, the second was the project that
I have chosen. 

A more descriptive title for my project now is ``Implementing Zero configuration
OSPFv3''. I shall be augmenting an existing implementation of the OSPF (Open
Shortest Path First) routing protocol to require no configuration on a simple
multiple subnet home network. The augmented implementation will be tested to
ensure that it works correctly both with homogeneous instances and instances of
alternative implementations. 

\section{Background}
As computing becomes more and more ubiquitous, home networks will have more and
more devices atached. It is difficult for us to predict all the possible uses of
network devices; some examples might include sensor networks to monitor the
home, network attached appliances such as ovens and dishwashers and home
surveilence systems (e.g. CCTV). 

With an increasing number of devices, the traffic on the network could become
very large quite quickly. This is a problem that could be easily reduced by
splitting the network into multiple subnets. For example one for sensors, one
for PCs and one for appliances. 

It would also be a convenient way of enabling public access to a home
network. A subnet could be created for guests to connect to, either over
wireless or some form of ethernet access point. The guest network could be
provided with some subset of the services of the main home network. Internet
access could be provided by SMB or NFS would still be unaccessable.

The IETF Home Networking (homenet) working group is doing a lot of exciting work
into the evolution of networking in residential homes. 

\section{Goals}

There are a number of goals that the project aims to meet, for it to be deemed
succesful.

\subsection{Project Goals }
The following goals are things that the project should do:

\begin{itemize}
	\item Allow multiple networks in the home. 
	\item Not require configuration: The user shouldn't need to enter any 
	settings, they should simply plug it in, and it should  work.
	\item Produce something of value for the community. The code I produce  
	should adhere to style and quality guidelines to ensure it is useful.
	\item Help to verify past implementations of the drafts.
	\item Discover any ambiguities in the drafts.
\end{itemize}

\subsection{Personal Goals}
By the end of the project I would like to have:

\begin{itemize}
	\item A better understanding of the process of the creation of internet
	standards.
	\item The ability to implement network protocols. 
	\item Improved programming skills (Mainly C or C++).
	\item Learned about new tools.
\end{itemize}

\chapter{Background and Literature Research}
The reading that I have done for this project can be split into two categories:
Theoretical reading which covers the theory behind the protocols, and Practical 
reading that covers the actual implementations to be used during the project.

\section{Theoretical Reading}
The majority of the information for this project is contained within various
RFCs (Request For Comments).  These are memos published by the IETF (Internet
Engineering Task Force) that are submitted for peer review.  RFCs form the
standards that are used across the Internet.

\subsection{RFC 2328} 
This RFC defines the OSPFv2 protocol, the protocol is a current internet
standard. It gives a complete description of the data structures and algorithms
of OSPF for IPV4. The author of this RFC J. Moy has also written two books that
continues the description. 

\subsection{RFC 5340}
This RFC defined the OSPFv3 protocol (OSPF for IPV6), the protocol is a
proposed standard.  Since OSPFv3 in general is fairly similar to OSPFv2,
the RFC describes only the differences between the two protocols. 

Neither Quagga nor XORP actually implement this RFC - Quagga implements
RFC 2740 which is technically compatible and XORP claims to implement
the draft of this RFC. 

\subsection{draft-dimitri-zospf-00}
This draft was published in 2002, and expired in 2003 - it never passed
version 00.  However, it lays out many of the main concepts for later
drafts.  A method was proposed for running a cut down version of OSPFv3
for routing both IPV4 and IPV6 traffic without requiring any manual
	configuration. 

\subsection{draft-ietf-ospf-ospfv3-autoconfig-00} 
This draft defines OSPFv3 Auto-Configuration, it is the main draft that will be
used for this project, it provides a set of modifications that can be made to
the OSPFv3 protocol to run (in a slightly restricted manner e.g.  single area)
it in without needing any manual configuration.  The draft discusses
configurations that should be set in some default manner, methods of selecting
Router-IDs (RID) and ways of detecting duplicate RIDs. 

The draft references many RFCs, a good understanding of the concepts 
that they convey is required to implement the draft successfully:

\begin{itemize}
	\item RFC 5838 - OSPFv3-AF (Address Families)
	\item RFC 6506 - OSPFv3-AUTH-TRAILER
	\item RFC 3630 (4230 \& 5786) - Traffic Engineering Extensions 
	\item RFC 4862 - SLAAC (IPv6) 
\end{itemize}


\subsection{draft-arkko-homenet-prefix-assignment-03}
 		
This draft is written by the same author as the previous draft, it defines
Prefix Assignment in Home Networks. as a testament to how recent this work is, a
new version of this draft has been released since the beginning of this project. 

\todo{Haven't really read this draft properly yet, won't summarise it
until I have a better understanding}


\subsection{Cisco's Guide to Configuring OSPFv3}

Although it is aimed at network administrators, this document provides a
nice high level overview of the OSPFv3 Protocol.



\subsection{OSPF: Anatomy of an Internet Routing Protocol}
 This book was written by the creator of OSPF (J. Moy). It defines the anatomy
 of OSPF in a greater level of detail than the RFC.

\subsection{OSPF: Complete Implementation}
The companion of the first book, also written by creator of OSPF. Gives a
complete implementation and walks you through it.

\subsection{OSPF and IS-IS}
Talks about chosing an IGP for large networks. Makes a feature by feature
comparison of OSPF with IS-IS.

\todo{Note how the books seem to predate OSPFv3}

\section{Practical Reading}		
Much of the practical information for this project is from online tutorials and
documentation of software projects.  

\subsection{Alix 2d3 Installation Guide}
This guide offered a comprehensive list of steps that need to be taken
to install Ubuntu linux on an Alix 2d3 board.  The article is written in
German but using Google translate it is quite easy to follow (since a
large amount of it is just commands).  

The process involves mounting the compact flash card on an existing Ubuntu
desktop.  The card is formatted and partitioned and a small install of Linux is
copied to the card using debootstrap.  Next, the file system root is set to
cards mount point and various configuration files are edited and essential
software packages (such as Vim, ssh and sudo) are installed. A boot-loader such
as grub is also installed.  Finally the Alix board is booted with the flash card
inserted while connected to a serial port.  {\bf cu} can be used to gain a
terminal connection to the board and ensure that there are no Magic ELF errors.

\subsection{OSPF Implementations}
There are many different implementations of the OSPF the three most popular open
source implementations are Xorp, Quagga and Bird. Each of the projects appear to 
be well documented. They all use Git as version control software and they all 
have somewhat active mailing lists (both user and developer).

\subsubsection{Xorp}
XORP is an implementation of various routing protocols, including
OSPFv3. It is Implemented in C++ following an Object Oriented style.
OSPFv2 and OSPFv3 share much of the same code with conditional code
separating the parts which are different.

\subsubsection{Quagga}

\todo{This seems a little opinionated, possibly make it more factual}

Quagga is another implementation of various routing protocols.  It is a
fork of the now defunct GNU Zebra project.  Quagga also implements
OSPFv3, however takes a more modular approach to doing so, providing a
completely seperate implementation of OSPFv2. The protocols are
implemented in C using a mainly Struct based style.  Vyatta the open
source software based virtual router notably switched from XORP to
Quagga. 

\subsubsection{BIRD}
BIRD is a more lightweight implementation. It also makes use of C
structs. It has already been used to successfully implement zOSPF.

\chapter{Design of System}
\todo{The justification of approach should be included in this section IMO.}
Much of the design of my project has been done by individuals and groups of
people who have far more knowledge and experience working with networks than I
do. 

\section{Chosen System}
The routing protcol suite that I have chosen to base my project on is Quagga. I
made this choice for many reasons. Firstly, having been subscribed to the
mailing lists for a few weeks now, I observed that Quagga has the most active
mailing lists. Secondly, Quagga splits up OSPFv2 and OSPFv3 - they do not share
any of the same source code; since I am only concerned with OSPFv3 for my
project, this should reduce some of the complexity involved with the
implementation. Quagga is used by the Vyatta project which gives it a large
amount of exposure. 

I also found it easiest to install set up and use Quagga. It has a precompiled
package in Ubuntu's apt (Advanced Packaging Tool) repositories. One of the
reasons that I decided against using BIRD is that an implementation for a new
platform is more helpful to the community. One potential downside to using
Quagga is that it uses plain struct based C, and I have more experience writing
object oriented code. 

\todo{Then go on to explain how Quagga is structured}

\section{Test Networks}
To ensure that my changes work as expected, I will need to test instances of it
running on a few typical test networks. 

\missingfigure{Test Network Topology}

\chapter{Tool and Techniques}
\todo{Are there really any Techniques in here?}

During the project I shall be making use of many different tools and techniques
to help me complete various tasks. One of my goals is to learn about new tools,
so I have tried to use tools that I only have a small amount of experience
working with. 

\section{Git}
I have chosen to use Git as my version control system. This choice was
partially influenced by the fact that Quagga uses Git, however there are many
other good reasons for using Git, or any other Distributed Version Control
System (DVCS). Firstly, version control is essential to a software project as
it enables you to keep track of changes over time and revert to old versions if
neccessary. Distributed version control is superior to traditional centralised 
version control system such as SVN as it allows work to be done without internet
connectivity and speeds up commit time (hopefully encouraging more frequent
commits). I shall be using bit bucket to back up my repository in the ``cloud''.

\section{Trello}
I shall be using trello as a task management platform. Trello is a web app based 
on the Kanban method of project manaement. It provides card than can be placed
into lists and moved around between lists. Lists can have titles such as
``Thinking'', ``Doing'', ``Done''. Trello will help me to keep track of what
needs doing during my project.

\section{JIRA}
JIRA is an issue tracking software. I shall use it to keep track of bugs and
features that I need to work on. 

\section{Jenkins}
Jenkins is a continuous integration server, it can be configured to pull your
code from a repository and build it to make sure it still works.  

\section{\LaTeX}
I have chosed to produce all of the documentation associated with this project
using \LaTeX. I chose \LaTeX because it is a powerful tool that allows me to
focus on writing my report rather than typesetting. It also tends to br more
robust for longer documents than traditional WYSIWYG editors such as Microsoft
Word.

\subsection{Todonotes Package}
I shall use the package todonotes to help me keep track of ideas and tasks that
need to be done in my document.

\subsection{Texcount}
As word counts are quite important throughout the project, I shall make use of
the texcount script to count the words in my documents.


\chapter{Report on Technical Progress}
So far, good progress has been made on the project.  The Alix 2d3 boards have
been set up to run Ubuntu Linux and have remained stable - both devices
currently have an uptime of 28 days.

Quagga has been installed on one of the routers, and XORP on the other.
Compilation of the projects on the machines themselves is very slow, so in the
future shall be done on a more powerful machine, for example a desktop PC.

All of the reading mentioned above has been read lightly, many of the documents
reference the other documents so require a more thorough understanding of the
concepts conveyed in the other documents.

RFC 2328 (OSPFv2) has been studied in great detail, progress has also been made
with studying RFC 5340 (OSPFv3). 

\chapter{Plan of Remaining Work}

\em

\textbf{
	By the time that this report is submitted, the following tasks should be
	complete (an as such moved to the above section) 
}

\begin{itemize}

\item Deeply read remaining RFCs
\item Get an instance of each of these protocols running on an Alix 2d3
\item Look at the source code of the various Implementations. 
\item Pick an Implementation. I have chosen to base my implementation on Quagga.
The reasons are as follows: 

\begin{itemize}
\item Quagga was far simpler to set up. I had many more difficulties installing
XORP on an ALIX 2d3.
\item Quagga is contained in the Ubuntu repositories, indicating that it is a
widely used package. 
\item Vyatta switched from XORP to Quagga, this shows that Quagga is a stable
enough platform to be trusted by third parties. 
\item As far as I know, no one else has implemented for this platform. This means
that my implementation will potentially be more useful to the community. 
\item Quagga makes a clear separation between OSPFv3 and OSPFv2. They are two
separate daemons with completely different source code. This is useful since the
project only concerns OSPFv3.
\end{itemize}


\end{itemize}

\em 

The following is a brief plan of how the project shall progress:

\begin{itemize}

\item Study the source code of the chosen implementation.

\begin{itemize}

\item Understand the chosen implementation.
UML diagrams could be drawn to summarise the architecture of the software. 

\item Understand the required changes.
The UML diagrams could be copied and augmented to show how the required
alterations will be realised. 

\end{itemize}

\item Perform the required changes to the chosen implementation.
\item Test the implementation with multiple instances of the same implementation.
\item Test the implementation with other implementations of protocol 
\item Test the implementation on people. \em Would require ethics approval etc. \em
\todo{Decide if I will test on people}
\item The project is open to extensions. Possible extensions are:

\begin{itemize}
\item Multiple Hop service discovery.
\item Source Based routing?
\end{itemize}

\end{itemize}

\section{Gantt Chart}
No project would be complete without a Gantt chart... I'll try and find the one
from Christmas dinner last year...

\missingfigure{Gantt Chart}

\chapter{Risk Analysis}

When using the power supplied, there is a risk that I could get an electric
shock. I took the power supplies to be PAT tested on Zepler level 2. As the
power supplied are double insulated (i.e. no single failure can result in a
shock), a quick visual inspection by a trained member of staff reassured me that
they would be safe to use. 

Network outage. On the 1st - 2nd (etc) Dec 2012 ECS had a fairly major network
outage due to a hardware failure. I was unable to do other coursework because I
needed to use the software within ECS. DCVS should help prevent this from being
an issue. Replicate code and resources in ECS, at home and in the cloud. 

\todo{When was the fire?}
There is a very real possibility of fire or some other unforeseeable damage to my
data, in 2005 the old Mountbatten building burned to the ground and a lot of
people lost work. To ensure that, if something like this happens, I will not be
affected, I shall make frequent backups of my work, using the aforementioned
source code control system.

As my project involves using specialist hardware, there is a possibility that
the hardware could break. Fortunately the operating system (Ubuntu) that is running on
the Alix boards is completely compatible with any x86 based computer. By adding
a second Network Interface Card to my desktop PC, I would be able to run the
software on there. 

It could turn out that I misjudged the difficulty of the project and I am not
able to complete in in the period of time available for a third year project.
Because of this I have chosen to structure my project in a modular way, with
different stages that I can attempt if I have time. This approach works quite
well as my project is fairly research based. 

\end{document}

