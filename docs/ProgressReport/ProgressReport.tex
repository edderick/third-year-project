\documentclass[12pt]{report}
\usepackage[utf8]{inputenc}
\usepackage{amsmath}
\usepackage{amsfonts}
\usepackage{amssymb}
\author{Edward Seabrook \\ 
Electronics and Computer Science \\ 
University of Southampton \\ \\
Project supervisor: Dr T Chown \\
Second examiner: Dr KP Zauner}
\title{Third Year Project Progress Report}

\usepackage{titlesec}
\titleformat{\chapter}% reformat chapter headings
    [hang]% like section, with number on same line
    {\Large\bfseries}% formatting applied to whole
    {\thechapter}% Chapter number
    {0.5em}% space between # and title
    {}% formatting applied just to title

 %Get rid of margins?
 \usepackage[top=2.4cm, bottom=2.4cm, left=3.5cm, right=2.4cm]{geometry} 

\usepackage{fancyhdr} 
\pagestyle{fancy} 
\renewcommand{\headrulewidth}{0pt} 
\lhead{}\chead{}\rhead{}
\lfoot{}\cfoot{\thepage}\rfoot{}

\begin{document}


\begin{titlepage}

\begin{center}


% Upper part of the page
%\includegraphics[width=0.15\textwidth]{./logo}\\[1cm]    

\LARGE Electronics and Computer Science\\
Faculty of Physical and Applied Sciences\\
University of Southampton
\\[1.5cm]

\href{mailto:ejfs1g10@ecs.soton.ac.uk}{Edward JF Seabrook}\\[0.5cm]

\today \\[1cm]
{\bfseries A Tool to Simplify Network Administration in the Modern Home}\\[1.5cm]

\vfill

% Author and supervisor
\large
Project Supervisor: 
Dr.~T \textsc{Chown}\\

\large
Secondary Examiner:
Dr.~KP \textsc{Zauner} 

\vfill

A Project Progress Report Submitted for the Award of Computer Science

\end{center}

\end{titlepage}



\begin{abstract}
\em No more than 200 words. \em


\end{abstract}

\tableofcontents

\chapter{Project Description}

The original project definition was to create a tool to simplify network
administration in the home.  This definition was left open ended to ensure that
the project was suitable before it was committed to.  The project has now been
more rigidly defined as ``Implementing Zero Config OSPF''.

An implementation of the OSPF (Open Shortest Path First) routing protocol shall
be augmented to require no configuration.

The augmented implementation will be tested to ensure that it works correctly
both with homogeneous instances and instances of alternative implementations. 

\section{Goals}

There are a number of goals that the project aims to meet.

\subsection{Project Goals }
The following goals are things that the project should do:

\begin{itemize}

	\item Multiple networks in the home. 

	\item No additional configuration:

		\begin{itemize}	
  		\item Shouldn't need to enter any settings, plug it in, it works. 
		\end{itemize}
\end{itemize}

\subsection{Personal Goals}
By the end of the project I would like to have:

\begin{itemize}
	\item Understanding of the process of the creation of internet standards.
	\item Ability to implement network protocols. 
	\item Improved programming ability (Mainly C/C++).
\end{itemize}

\chapter{Background and Literature Research}

The reading for this project can be split into two categories: Theoretical
reading which covers the theory behind the protocol that shall be implemented,
and Practical reading that covers the actual implementations that shall be used
during the project.

\section{Theoretical Reading}
The majority of the information for this project is contained within various
RFCs (Request For Comments).  These are memos published by the IETF (Internet
Engineering Task Force) that are submitted for peer review.  RFCs form the
standards that are used across the Internet.

\begin{itemize}

	\item RFC 2328 - OSPFv2 

	This RFC defines the OSPFv2 protocol, the protocol is a current 
	internet standard. 
	It gives a complete description of the data structures and algorithms
	of OSPF for IPV4. 
	The author of this RFC J. Moy has also written two books that 
	continues the description. 

	\item RFC 5340 - OSPFv3
	
	This RFC defined the OSPFv3 protocol (OSPF for IPV6), the protocol is a
	proposed standard.  Since OSPFv3 in general is fairly similar to OSPFv2,
	the RFC describes only the differences between the two protocols. 

	Neither Quagga nor XORP actually implement this RFC - Quagga implements
	RFC 2740 which is technically compatible and XORP claims to implement
	the draft of this RFC. 

	\item draft-dimitri-zospf-00 - Zeroconf OSPF

	This draft was published in 2002, and expired in 2003 - it never passed
	version 00.  However, it lays out many of the main concepts for later
	drafts.  A method was proposed for running a cut down version of OSPFv3
	for routing both IPV4 and IPV6 traffic without requiring any manual
		configuration. 

	\item draft-ietf-ospf-ospfv3-autoconfig-00 - OSPFv3 Auto-Configuration 
	
	The main draft that will be used for this project, it provides a set of
	modifications that can be made to the OSPFv3 protocol to run (in a
	slightly restricted manner e.g.  single area) it in without needing any
	manual configuration.  The draft discusses configurations that should be
	set in some default manner, methods of selecting Router-IDs and ways of
	detecting duplicate R-IDs. 

	The draft references many RFCs, a good understanding of the concepts 
	that they convey is required to implement the draft successfully:

	\begin{itemize}
			\item RFC 5838 - OSPFv3-AF (Address Families)
			\item RFC 6506 - OSPFv3-AUTH-TRAILER
			\item RFC 3630 (4230 \& 5786) - 
			Traffic Engineering Extensions 
			\item RFC 4862 - SLAAC (IPv6) 
			\end{itemize}
	

	\item  draft-arkko-homenet-prefix-assignment-03 -
	Prefix Assignment in Home Network
 		
	This draft is written by the same author as the previous draft, and as a
	testament to how recent this work is, a new version of this draft has
	been released since the beginning of this project. 
	
	\em Haven't really read this draft properly yet, won't summarise it
	until I have a better understanding \em


	\item Cisco's Guide to Configuring OSPFv3 
	
	Although it is aimed at network administrators, this document provides a
	nice high level overview of the OSPFv3 Protocol.

	\item The Three Books

	\em Will change this point \em 
	
	Yet to read cover to cover - have used briefly for reference. 

\end{itemize}

\section{Practical Reading}		

Much of the practical information for this project is from online tutorials and
documentation of software projects.  

	\begin{itemize}

	\item Alix 2d3 Installation Guide

	This guide offered a comprehensive list of steps that need to be taken
	to install Ubuntu linux on an Alix 2d3 board.  The article is written in
	German but using Google translate it is quite easy to follow (since a
	large amount of it is just commands).  The process involves mounting the
	compact flash card on an existing Ubuntu desktop.  The card is formatted
	and partitioned and a small install of Linux is copied to the card using
	debootstrap.  Next, the file system root is set to cards mount point and
	various configuration files are edited and essential software packages
	(such as Vim, ssh and sudo) are installed. A boot-loader such as grub is
	also installed.  Finally the Alix board is booted with the flash card
	inserted while connected to a serial port.  {\bf cu} can be used to gain
	a terminal connection to the board and ensure that there are no Magic
	ELF errors. 

	\em Not sure whether the following strictly belongs in a reading section
	\em

	\item Xorp
	
	XORP is an implementation of various routing protocols, including
	OSPFv3.  It is Implemented in C++ following an Object Oriented style.
	OSPFv2 and OSPFv3 share much of the same code with conditional code
	separating the parts which are different.
	
	\item Quagga
	
	Quagga is another implementation of various routing protocols.  It is a
	fork of the now defunct GNU Zebra project.  Quagga also implements
	OSPFv3, however takes a more separated approach to doing so, providing a
	completely seperate implementation of OSPFv2. 	The protocols are
	implemented in C using a mainly Struct based style.  Vyatta the open
	source software based virtual router notably switched from XORP to
	Quagga. 

	\item Bird
	Bird is a more lightweight implementation.  It has already been used to
	successfully implement zOSPF.

\end{itemize}

	All three implementations appear to be well documented.  All use Git as
	version control software.  All have somewhat active mailing lists
	(Developer and user)


\chapter{Report on Technical Progress}

So far, good progress has been made on the project.  The Alix 2d3 boards have
been set up to run Ubuntu Linux and have remained stable - both devices
currently have an uptime of 28 days.

Quagga has been installed on one of the routers, and XORP on the other.
Compilation of the projects on the machines themselves is very slow, so in the
future shall be done on a more powerful machine, for example a desktop PC.

All of the reading mentioned above has been read lightly, many of the documents
reference the other documents so require a more thorough understanding of the
concepts conveyed in the other documents.

RFC 2328 (OSPFv2) has been studied in great detail, progress has also been made
with studying RFC 5340 (OSPFv3). 

\chapter{Plan of Remaining Work}

\em

\textbf{
	By the time that this report is submitted, the following tasks should be
	complete (an as such moved to the above section) 
}

\begin{itemize}

\item Deeply read remaining RFCs
\item Get an instance of each of these protocols running on an Alix 2d3
\item Look at the source code of the various Implementations. 
\item Pick an Implementation. I have chosen to base my implementation on Quagga.
The reasons are as follows: 

\begin{itemize}
\item Quagga was far simpler to set up. I had many more difficulties installing
XORP on an ALIX 2d3.
\item Quagga is contained in the Ubuntu repositories, indicating that it is a
widely used package. 
\item Vyatta switched from XORP to Quagga, this shows that Quagga is a stable
enough platform to be trusted by third parties. 
\item As far as I know, no one else has implemented for this platform. This means
that my implementation will potentially be more useful to the community. 
\item Quagga makes a clear separation between OSPFv3 and OSPFv2. They are two
separate daemons with completely different source code. This is useful since the
project only concerns OSPFv3.
\end{itemize}


\end{itemize}

\em 

The following is a brief plan of how the project shall progress:

\begin{itemize}

\item Study the source code of the chosen implementation.

\begin{itemize}

\item Understand the chosen implementation.
UML diagrams could be drawn to summarise the architecture of the software. 

\item Understand the required changes.
The UML diagrams could be copied and augmented to show how the required
alterations will be realised. 

\end{itemize}

\item Perform the required changes to the chosen implementation.
\item Test the implementation with multiple instances of the same implementation.
\item Test the implementation with other implementations of protocol 
\item Test the implementation on people. \em Would require ethics approval etc. \em

\item The project is open to extensions. Possible extensions are:

\begin{itemize}
\item Multiple Hop service discovery.
\item Source Based routing?
\end{itemize}

\end{itemize}

\section{Gantt Chart}
No project would be complete without a Gantt chart... I'll try and find the one
from Christmas dinner last year...

\end{document}
