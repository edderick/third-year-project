\appendix 
%\appendixpage
\chapter{OSI Model}
\label{osi}
  \begin{center}
  \begin{tabular}{|lc|}
  \hline
  7: &
  Application Layer \\
  & (e.g.\ DNS, DHCP) \\
  \hline
  6: &
  Presentation Layer \\
  & (e.g.\ telnet) \\ 
  \hline
  5: &
  Session Layer \\
  & (e.g.\ RPC) \\
  \hline
  4: &
  Transport Layer \\
  & (e.g.\ TCP, UDP) \\
  \hline
  3: &
  Network Layer \\
  & (e.g.\ IPv4, IPv6) \\
  \hline  
  2: &
  Data Link Layer \\
  & (e.g.\ MAC) \\ 
  \hline  
  1: &
  Physical Layer \\
  & (e.g.\ Ethernet, Wi-Fi) \\
  \hline
\end{tabular}
\end{center}

The seven layer OSI model is a logical grouping of the types of protocols found
in computer networks. The protocols in each layer tend to dependend only on the
protocols in the layer layer directly below them. 

The TCP/IP model is a similar, but simpler model of networking.

\chapter{Additional zClient-zServ Messages}
\label{zClient}
I extended the zClient-zServ protocol to expose some of the features,
previously only available through \texttt{vtysh}, to the routing protocol
implementations. 

\begin{center}
	\includegraphics[width=0.9\linewidth]{../Diagrams/Packets/nd_no_suppress.png}
\end{center}

The first of these messages is one to enable sending Router Advertisements. It
works by stopping NDP messages from being suppressed. There is also a
corresponding message to turn these RAs off again. 

\begin{center}
	\includegraphics[width=0.9\linewidth]{../Diagrams/Packets/ipv6_nd_prefix.png}
\end{center}

The next message adds a network prefix to the RAs. This is used to advertise a
new assigned prefix. There is also a similar message to remove the prefix.


\begin{center}
	\includegraphics[width=0.9\linewidth]{../Diagrams/Packets/addr_add.png}
\end{center}

The final message adds an IPv6 Address to a specified interface. In my
implementation this was used to associate a prefix with an interface for the
routing table calculations. Again, there is also a message to remove the
address.

\chapter{Netkit}
\begin{center}
	\includegraphics[width=\linewidth]{../Diagrams/Netkit/NetkitScreenshot.png}
\end{center}

Above is a screen shot of a typical Netkit session.  

Netkit instances are started with \texttt{vstart \$NAME}.

A lab is a collection of VMs and collision domains, described in a
configuration file. Labs are started with \texttt{lstart}. It is often useful
to allocate the VMs in the lab more memory using the flag \texttt{-o-M}n, where
n is the size of the memory to allocate in megabytes.

\chapter{radvdump}
\label{radvdump}
This \texttt{radvdump} output shows the contents of the router advertisements
made by one of my quagga instances. The prefix fc35:6225:4b8:6::/64 is a ULA
that was generated automatically by my implementation.

\lstinputlisting{../Issues/radvdump.md}

\chapter{GitHub Stats}
\label{GithubStats}
GitHub offers various graphics that can be used to visualise the progress and
patterned in the project. 

\section*{GitHub Commit Graph}
\begin{center}
	\includegraphics[width=\linewidth]{../Diagrams/Stats/GitHubCommitGraph.png}
\end{center}

This commit graph shows all of the commits that I made to my fork of Quagga.
The graph shows that I began working on my implementation in early February and
worked steadily, peaking at the beginning of the Easter vacation. The second
peak is when I began working on testing.

\section*{GitHub Punchcard}
\begin{center}
	\includegraphics[width=\linewidth]{../Diagrams/Stats/GitHubPunchCard.png}
\end{center}

This punch card shows the time of day, and day of the week that I made 
my contributions. The graph is for my documentation repository as the code
repository contained many commits by other authors.

\chapter{Alix2d3}
As the project progressed it became clear that they Alix2d3s were not as useful
as was originaly believed, thanks to discovery of the briliant software package
Netkit. I did however undertake testing on the Alix2d3s. If someone wished to
repeat my experimentation, the following guides would be helpful. 

\section*{Ubuntu Installation Guide}

 This guide\cite{germanGuide} offered a comprehensive list of steps that need to
 be taken to install Ubuntu Linux on a PC Engines alix2d3\cite{alix2d3}.  The article is
 written in German but using Google translate I was able to follow it as it is
 mainly a list of shell commands.

 The process begins with installing the compact flash card as a device on an
 existing Ubuntu desktop PC\@. Next the card is partitioned using \texttt{fdisk},
 the card is then formatted as ext2 (a Linux file system type) using
 \texttt{mk2fs}. The card is then mounted in the file system, using
 \texttt{mount} and a small installation of Linux is copied to the card using
 \texttt{debootstrap} to provide a base system.

 Several settings and devices are linked to the flash cards mount point and
 \texttt{chroot} is used to simulate booting into the new install. The required
 configuration files are edited (e.g.\@ network interface settings) and essential
 software packages (such as \texttt{Vim}, \texttt{SSH}, \texttt{sudo}, and
 \texttt{APT}) are installed. A boot loader such as \texttt{GRUB} is also
 installed and configured.

 Next, the flash card is safely removed, and  inserted into the alix2d3. The
 alix2d3 is then powered on. Using a USB to Null Model cable connecting a PC to
 the alix2d3's serial port, a terminal connection can be established using {\bf
 cu}. I had a few problems using this device (\texttt{ttyUSB0}) but I was able to
 fix this problem using \texttt{chown}. This terminal connection can be used to
 aid the boot process and ensure that there are no Magic ELF errors. Once the
 alix2d3 is up and running it is possible to disconnect the serial cable and
 access it over \texttt{SSH} alone.

 \section*{Quagga Installation Guide}

 \subsection*{Normal Installation}
This procedure turned out to be far more difficult that anticipated. Firstly,
all the dependencies must be installed. The easiest way I have found to do this
by using:

\texttt{sudo apt-get build-dep quagga}

This should fetch all the packages that Quagga depends on. These do not need to
be built from source, since I have not modified them.\ \texttt{libtool} and
\texttt{libreadline} need to be installed separately. 

Next it should be possible to run \texttt{\@./bootstrap}, this creates the
configure script; This can be run with the command:

\begin{quote}
\texttt{sudo \@./configure \\ --sysconfdir=/usr/local/quagga \\ --localstatedir=/usr/local/quagga}
\end{quote}

The configure script will prompt you to create a user called quagga:

\begin{quote}
\texttt{sudo adduser quagga} \\
\texttt{sudo mkdir /usr/local/quagga} \\
\texttt{sudo chown quagga:quagga /usr/local/quagga} 
\end{quote}

It is now time to run \texttt{make}. See the next section for more information
about compiling for the Alix2d3. I issues with making zebra which I fixed
by manually editing the Makefile in \texttt{\@./zebra}, adding \texttt{-lcap}
to \texttt{LIBS}\@. This issue no longer occurs with the latest release of
Quagga.

Running \texttt{sudo make install} should install everything -- the binaries for the daemons
should be located in \texttt{/usr/local/sbin}.

Finally a little bit of configuration is required. \path{/usr/local/quagga}
contains the configuration files, either make a real config file for the
daemons (zebra and ospf6d), or just create file containing ``password test''.

You should now be able to run \texttt{ospf6d}. If you can't, try \texttt{ldconfig}.

\subsection*{Cross Compilation}
\label{cross_compile}
Since Alix2d3s are very slow, building Quagga can be incredibly slow. The make
parts alone can take around 15 minutes. (XORP had an even slower compile time,
in the order of hours) 

To overcome this issue it is possible to compile the source code on a much
faster machine (e.g.\@ desktop pc) and copy it across. 

Since modern machines use a 64bit architecture rather than 32bit like the
Alix2d3s, the compiler must be set up correctly. The cflag \texttt{-m32
-march=i386} can be used. This can be done by appending
\texttt{--with-clfags=`-m32 -march=i386i'} to \texttt{\@./configure}.

I initially had problems with this compilation. I resolved the issues by going
back to basics and compiling a simple ``helloworld'' program for the
Alix.  Turns out I was missing \texttt{gcc-multilib}. This can be installed
with \texttt{apt}.

\chapter{vtysh}
\label{vtysh}
Commands are added using the macro \texttt{DEFUN} and the function \texttt{install}.

I added the commands:
\begin{itemize}
	\item \texttt{show ipv6 ospf6 prefix aggregated} \\
		Used to display all aggregated prefixes in the system.
	\item \texttt{show ipv6 ospf6 prefix allocated} \\ 
		Used to display the only aggregated prefixes that were allocated to this
		router only.
	\item \texttt{show ipv6 ospf6 prefix assigned X} \\
		Used to display the prefixes that have been assigned to a given interface
		by the prefix assignment algorithm. 


	\item \texttt{enable $\rightarrow$ configure terminal 
		$\rightarrow$ ipv6 allocate-prefix X:X::X:X/M} \\
		Used to manually allocate a new aggregated prefix to the routers.
	\item \texttt{enable $\rightarrow$ configure terminal
		$\rightarrow$ no ipv6 allocate-prefix X:X::X:X/M} \\
		Used to remove a previously allocated aggregated prefix.
\end{itemize}
\pagebreak

\begin{landscape}
\chapter{Structure of ospf6d}
\label{ospf6d}
\begin{center}
  \includegraphics[width=0.9\linewidth]{../Diagrams/UML/quaggaOSPF6D/ospf6d_simple.png}
\end{center}

\end{landscape}

\chapter{Test Results}
\label{testResults}
Of the twenty test cases that I produced, all of them now pass. The following is an
example of the output of my unit testing code: 

\begin{lstlisting}
%\ldots%
Testcase 5  %{\color{green} OK}%
Toal interface count: 2 
 Aggregated Prefix: fc00::/48
 Aggregated Prefix: fc00:1::/48
Total Aggregated Prefix count: 2 
 eth0's Assigned Prefix count: 2
  Assigned Prefix: fc00:0:0:1::/64
   Not Valid
   Deprecation Thread
  Assigned Prefix: fc00:0:0:3::/64
 eth1's Assigned Prefix count: 2
  Assigned Prefix: fc00:0:0:2::/64
  Assigned Prefix: fc00:1::/64
   Pending Thread
Total Assigned Prefix count: 4 
%\ldots%
Total Failed Tests: 0
\end{lstlisting}

\chapter{File Listings}
As this project is based on an existing software project, it would be
unreasonable to list all of the files contained in my submitted archive.
Instead, the following is a list of the most notable files: Those that I have
created, modified or made heavy use of throughout the project. The best way to
see what I changes I have made is to view my project on
GitHub.\footnote{\url{https://github.com/edderick/quagga_zOSPF}}

\section*{Netkit Labs}
The folder \path{netkit} contains a series of Netkit labs. I created these lab
folders to contain the settings of Netkit Labs.

In each lab folder, \path{lab.conf} contains the definition of the collision domains.
When a VM boots, the file consisting of its name suffixed with \path{.startup}
is run, and the files within the folder structure rooted at the folder with its
name are copied to its root directory.

\section*{Quagga}
The majority of the work that I have done is contained within
\path{quagga_zOSPF/ospf6d/}. I made modifications to many of the file in this
folder. \path{ospf6_auto.h} and \path{ospf6_auto.c} are my own additions. I
also made substantial additions to \path{ospf6_intra.h} and
\path{ospf6_intra.c}.

I also made alterations to several files in \path{quagga_zOSPF/lib/}. Most
notably: \path{prefix.h}, \path{prefix.c}, \path{zclient.h} and
\path{zclient.c}. In \path{quagga_zOSPF/zebra/}, I modified \path{zserv.c}.

My unit testing code can be found in \path{quagga_zOSPF/tests/}, contained
entirely within the file \path{ospf6d_autoconf_test.c}. When compiled, this
generates the file \path{testospf6dautoconf}.

\section*{hnet-core.diff}
The file \path{hnet-core.diff} contains the changes that I made to hnet-core,
Markus' autoconf OSPFv3 implementation, to increase its compatibility with my
own.

\chapter{Original Project Brief}

\section*{A Tool to Simplify Network Administration in the Modern Home}

\subsection*{Problem}

As computing becomes more and more ubiquitous, home networks are becoming
increasingly complex. In the future we may see households containing a very
large number of devices, performing functions we can't yet even imagine. 

As home networks grow in both number of hosts and bandwidth requirements, the
task of administering such networks will inevitably become more complicated. 

Unfortunately, most homeowners don't have degrees in computer science, so will
find it hard to set up effective networks in their homes. 

\subsection*{Goals}

The project's main aim is to simplify the administration of complex home
networks. This goal shall be realised by providing users with a system that
either reduces the complexity of managing a complex, multi-subnet, home
network, or providing tools for monitoring the activity of the network. 

\subsection*{Scope}


The project will involve either:
\begin{itemize}
\item Implementing a zeroconf OSPF routing system based on an open source
  routing implementation such as Quagga or XORP.

\item Create a traffic monitoring system that can report on bandwidth usage,
  network flows and perform some form of malware detection. With the
  possibility of implementing a front-end to allow easy QoS configuration. 

\end{itemize}

Both of these projects will involve the use of lightweight Linux boxes to
perform their tasks. The project that will be pursued will be decided upon by
week four. 


