\documentclass[12pt]{report}
\usepackage[utf8]{inputenc}
\usepackage{amsmath}
\usepackage{amsfonts}
\usepackage{amssymb}
\author{Edward Seabrook} 
\title{Third Year Project Final Report}

\usepackage{nomencl}
\makenomenclature

% Makes Chapter heading look like Section heading
\usepackage{titlesec}
\titleformat{\chapter}% reformat chapter headings
    [hang]% like section, with number on same line
    {\Large\bfseries}% formatting applied to whole
    {\thechapter}% Chapter number
    {0.5em}% space between # and title
    {}% formatting applied just to title

% Save a bit of space by giving all headings less room
\titlespacing*{\chapter}{0pt}{0pt}{0pt}
\titlespacing*{\section}{0pt}{0pt}{5pt}
\titlespacing*{\subsection}{0pt}{0pt}{5pt}
\titlespacing*{\subsubsection}{0pt}{0pt}{5pt}
\titlespacing*{\paragraph}{0pt}{0pt}{5pt}

% Set margins to required
\usepackage[top=2.4cm, bottom=2.4cm, left=3.5cm, right=2.4cm]{geometry} 

% Sort out margins for todonotes
\setlength{\marginparwidth}{3cm}
\reversemarginpar

% Set paragraph spacing to required
\setlength\parindent{0pt}
\usepackage[parfill]{parskip}

\usepackage{hyperref}
\usepackage{todonotes}
\usepackage{listings}
\usepackage{appendix}
\usepackage{pdflscape}
\usepackage{cite}

% Not totally sure
\usepackage{fancyhdr} 
\pagestyle{fancy} 
\renewcommand{\headrulewidth}{0pt} 
\lhead{}\chead{}\rhead{}
\lfoot{}\cfoot{\thepage}\rfoot{}

% Number and show in ToC to a deeper level
\setcounter{secnumdepth}{3}
\setcounter{tocdepth}{3}

\def\nomlabel#1{\textbf{#1}\hfil}


\begin{document}

% Include title page

\begin{titlepage}

\begin{center}


% Upper part of the page
%\includegraphics[width=0.15\textwidth]{./logo}\\[1cm]    

\LARGE Electronics and Computer Science\\
Faculty of Physical and Applied Sciences\\
University of Southampton
\\[1.5cm]

\href{mailto:ejfs1g10@ecs.soton.ac.uk}{Edward JF Seabrook}\\[0.5cm]

\today \\[1cm]
{\bfseries A Tool to Simplify Network Administration in the Modern Home}\\[1.5cm]

\vfill

% Author and supervisor
\large
Project Supervisor: 
Dr.~T \textsc{Chown}\\

\large
Secondary Examiner:
Dr.~KP \textsc{Zauner} 

\vfill

A Project Progress Report Submitted for the Award of Computer Science

\end{center}

\end{titlepage}


\begin{abstract}
As home networks become more and more complex, it is inevitable that they will
require splitting into multiple subnets. As the average home user is unable to
configure a router, a minimal configuration solution is required. In this
project I produced an implementation zero configuration OSPF based on drafts
published by the IETF\@. 
\end{abstract}

\tableofcontents
\clearpage

\chapter{Acknowledgements}

\begin{itemize}
\item Tim Chown - Offered support as project supervisor
\item David Lamparter - Responded to my queries about how Quagga works
\item Jari Arkko - Wrote the drafts that I was implementing
\item Markus Stenberg \& Benjamin Paterson - Provided implementations against which 
      to interop test
\item Freinds \& Family for proof reading this document.
\end{itemize}

\chapter{Introduction}
\todo{A clear statement of the problem and goals of the project.}
The title of my project is ``Implementing Zero Configuration OSPFv3''. During
the project, I have modified an open source implementation of OSPFv3, to comply
with two recently published internet drafts. The changes allow routed networks
to be set up without the need for manual configuration.

\section{The Problem}
As computing becomes more and more ubiquitous the complexity of an average home
network will increase. Not only will the number of devices rise, but with the
advent of sensor networks and home automation, the number of types and
topologies of these networks will grow. To ensure these more complex networks
run smoothly, networks must be separated into multiple subnets.

Unfortunately, the vast majority of home networks do not have a professional
System Administrator around to configure and maintain them. This means ideally
a network shouldn't require any configuration to work.

A less relevent problem is \nomenclature{IPv4}{Internet Protocol Version 4}
address space exhaustion. 

\section{The Goals}
This project set out to meet many goals, over the course of the project, the
goals transformed slightly as my understanding of the problem improved. The
goals for the project were:
\todo{Update the goals}
\begin{itemize}
\item Allow multiple subnets in the home; allow hosts to communicate from one
  subnet to another.
\item Require no configuration: The user should not need to enter any settings,
  they should simply plug in the router, and it should work.
\item Produce something of value for the community. The code I produce should
  adhere to style and quality guidelines to ensure it is useful.
\item Verify past implementations of the drafts.
\item Discover any ambiguities in the drafts.
\end{itemize}

==Emphasise bleeding edge?==

\chapter{Background}
\todo{Is this section a big waffle?}
To enable me to explain my project properly, I first need to explain concepts
that it relies on.

\section{Home Networks}
Modern home and small/home office (SOHO) networks all tend to follow the same
patterns. A single ISP provides a connection to the internet using the phone
line or television cable.  The ISP will typically provide a single IPv4 address
(/32), usually this is an unstable (dynamic) allocation although a static IP is
often offered as a premium service. 

To allow multiple hosts on the network, Network Address Translation (NAT) is
usually employed. With NAT, the router rewrites the address of the packets from
the globally allocated IP address to a local address often in the subnet
192.168.1.0/24. The port number is often used to multiplex between the
different hosts on the network. To allow applications like Skype to function
Universal Plug and Play (UPnP), a technlogoy that allow automatic port mapping
on a network, is often used. IP addresses are delegated to the hosts of the
local network using the Dynamic Host Configuration Protocol (DHCP). 

The solution is usually provided as a single item of Customer Premises
Equipment (CPE), refered to by many names, most commonly as a ``Router''.
This device provides the modem for connecting to the
internet, an Ethernet switch with around 4 ports, and a wireless access point
(802.11x) that is usually bridged to the wired network to provide just one
subnet. 

\section{Campus and Enterprise Networks}
Larger networks like the ones found on University Campuses and Larger
Businesses tend to employ many of the more advanced features of networking.
They are often split into many subnets depending on the physical locations,
workgroups and expected use of the hosts. 

Many organisations were able to obtain large chunks of IPv4 address space, the
University of Southampton was able to obtain a /16 (152.78.0.0) giving them
around 65000 addresses to play with. Having this many addresses allows each
machine on the network to be globally addressable - people's initial reaction
to this is often that it would be insecure, but a good firewall is far more
effective protection than NAT. Even with this many addresses, the University
still struggles. This is due at part to address space wastage, the now
deprecated ISS wireless service still takes up a large proportion of the
address space.

Although it is not yet a common sight, the networks run by ECS are dual stack,
meaning the machines are allocated both IPv4 and IPv6 addresses. This
demonstrates than IPv6 is mature enough to be deployed in large networks.

It should be noted however, that at the moment, to run a network the size of
the University's, many specialist staff are required to configure the network
and ensure its day to day running. This is something that the vast majority of
households are not able to obtain.

\section{Future Networks}
In the future users will want more from their home networks. We are likely to
see a dramatic increase in the number of devices in the average home, ranging
from more laptops and tablets, to more sensors and home automation devices.
As these types of traffic may be heterogeneous, it would be beneficial to split
the network up into multiple subnets to ensure the best performance from each
network. 

Guest networks are already an option provided by many high end wireless access
points. By using Virtual Access Points (VAPs) guest clients can connect to the
internet, but are restricted from having full access to the local network. To
facilitate this, hard coded local IPv4 addresses are usually used, with the
advent of IPv6 and globally valid addresses, a mechanism to assign IP addresses
and route traffic is required.  

\todo{Stop. Elaborate and Listen}
The Internet Engineering Task Force's (IETF) Homenet working group
\cite{homenet} are working on the architecture of the home network of the
future. 

\section{IPv6}
To comminicate with a computer, we need a way to uniquely identify it. At
present, IPv4 is the dominating technology for addressing computers on the
internet. Unfortunately when IPv4 was designed, it was not forseeable that we
would need more than 4 billion addresses, so a neat 32 bit identifier was used.
In 2011 however, The Internet Assigned Numbers Authority (IANA) ran out of IPv4 
addresses. 

One solution to the IPv4 address space exhaustion problem is migration to
IPv6; a protocol that is fundamentally similar to IPV4, but instead uses 128
bit addresses. This gives us far more addresses to work with - in fact a total
of $3.4028237\times10^{38}$ unique addresses. With estimates for the total
number of grains of sand on Earth being around $7.5\times10^{18}$, the number
of stars visible from Earth falling at about $7.0\times10^{22}$, and more
relevently the estimated human population of Earth being $7.0\times10^{9}$,
this address space should be sufficient for atleast the forseeable future. 

IPv6 address ranges are often given in (Classless Inter-Domain Routing) CIDR 
notation. In this notation an IPv6 address is written, followed by a slash and then 
and number between 0 and 128. This number represents the number of bits in the IPv6 
address that idenitfy the network, and the rest of the bits idenitfy the host.

IPv6 prefixes tend to be /64, as this is required by Stateless Address 
Autoconfiguration (SLAAC). A /48 prefix may be allocated by an ISP to allow the user 
to assign multiple subnets on their network.

\section{Routing}
For packets of data to find their way around networks, they require
routing. The routing we are concerned with is Layer 3 routing. When an IP
packet is sent, it's address is compared with those stored in the kernel's
routing table, and then transmitted using the appropriate interface. Networks
contain many computers whose sole purpose is to forward network traffic, the
computers are known as routers. 

A number of different protocols exist to build up these routing tables:

\subsection{Interior gateway protocols}
Interior Gateway Protocols (IGPs) are concerned with building the routing
tables for a single autonomous system (AS). 

\subsubsection{Distance Vector}
Distance vector routing works by having each router advertise to its neighbors
(Directly connected routers) the shortest path it knows about to each subnet.
Although less complex that Link-state routing Protocols, they are often slower
to converge, and experience what is know as the ``count to infinity'' problem
whereby loops in the network topology can lead to weights increasing to
infinity when a router goes down. 

The most common Distance Vector protocol is Routing Information Protocol (RIP),
a routing protocol famous for defining 15 as infinity. IGRP and EIGRP are
Cisco's own propriety alternatives.

\subsubsection{Link-State}
Link state routing protocols work by allowing each router on the network to get
a full understanding of the network topology. From this representation of the
networks topology, the router then performs some calculations to work out the
best route. The two most common Link-State routing protocols, OSPF and IS-IS,
both use Dijkstra's algorithm to calculate these shortest paths. 

OSPF and IS-IS are both very similar protocols, the main difference is that
IS-IS is as a Layer 2 protocol is Layer 3 agnostic, whereas OSPF is depended on
IPv4 or IPv6 depending on the version.

\subsection{Exterior Gateway Protocols}
In this project I was not concerned with Exterior Gateway Protocols (EGP), the
main example being Border Gateway (BGP). These kinds of protocols do routing
based on policy than the technical distance of the destinations.

\chapter{Literature Review} 
\todo{Could go on forever. Need to be concise.}

The project was based mainly on two closely related drafts, both written by the same
authors. If these drafts are approved by the IETF they will end up becoming Request 
for Comments (RFC) which are the technical documents that specify how the internet 
works. The most notable RFCs are placed on a Standards Track, and if they are 
successful will become recognised as a proposed standard. Eventually proposed 
standards may become Internet standard and are seen as the highest form of standard 
the IETF produces. 

\section{OSPFv3 Autoconfiguration}
This draft describes how OSPF for IPv6 can be extended to run without requiring
configuration. The draft begins by enforcing some default values specified in the
original OSPF RFCs, and restricting a few features of OSPF such as allowing only the 
backbone area to be used. It then goes on to specify a mechanism for generating 
Router IDs and ensuring that there are no conflicts across the network. This is 
achieved by the introduction of a new Link State Advertisment (LSA) type.

\section{OSPFv3 Prefix Assignment}
Building on the work of the Autoconfiguration draft, this draft solves
the issue of delegating IPv6 network prefixes from a larger pool of IPv6 
prefixes. As well as offering a realistic solution to a problem, the draft 
also demonstrates the extensibility of the Autoconfiguration draft. 

A method for breaking up a short prefix delegation (possibly a /48 issued by
ISP) into many longer (/64) prefixes is given, along with a strategy for avoiding conflicts in the assignment
across the network.

\section{Other Important Drafts}
There are several other documents that are relevant to this project. These are
both articles that are relevant as background knowledge and papers that have
been referenced from one another.

\subsection{Home Networking Architecture for IPv6} 
[draft-ietf-homenet-arch-07] puts forward a vision of future home networking. The 
draft puts forward a complete architecture for future home networks, and goes on to 
give a discussion of many emerging technologies. 

\subsection{OSPF Version 2}
I studied [RFC2328] in depth to understand how the OSPF Protocol works. The 
memo is surprisingly easy to read, and offers a complete discussion of how the 
OSPF protocol is designed to build routing tables for IPv4 . 

Although the document contains many details that are irrelevent to this project, 
including a long discussion on areas, and details of the SPF algorithms

\subsection{OSPF for IPv6}
[RFC5340] defines OSPFv3. The RFC builds on RFC2328, explaining the differences between OSPFv2 and OSPFv3. 

\subsection{Autoconfiguration of Routers Using a Link State Routing Protocol}
[draft-dimitri-zOSPF] - An old draft that had similar aims to the OSPFv3 
Autoconfiguration draft. This draft focuses on allowing autoconfiguration for both 
IPv4 and IPv6 rather than just IPv6; this made it a more complex and restrictive 
soltion. 

\subsection{The OSPF Opaque LSA Option}
RFC5250 introduces the opaque LSA. An opaque LSA is a generic type of LSA that can 
be used to transmit application specific data. The concept is similar to that used 
in the autoconfiguration draft, but it does not make use of Type-Length-Values 
(TLVs). I found this draft useful mostly because of its brief discussion about the  
reachability of LSAs.

\section{Books}
\todo{List the books}
There were a few books that were relevant to this project. In general the books were 
useful for clarifying things that were not very clear from the RFCs. They were also 
useful for reinforcing my general understanding of the protocol, as they gave a 
similar level overview, written from a slightly different angle. 

Unfortuately the books didn't cover much OSPFv3 because it is a relatively recent 
development, and there was no mention atall of more recent drafts (such as the 
autoconfig draft) because they are very much on the cutting edge.

\chapter{Analysis}
An analysis and specification of the solution to the problem

\section{Alternatives}
There are also other approaches that could be used to solved the same problems as 
these drafts. The different approaches all have thier own pros and cons, but to get 
a full understanding of the problems that occur with each of the solutions, it is 
neccesary to produce implementations of each soltution.

\subsection{IPv4 Address as Router ID}
A possible solution to the Router ID assignment probem would be simply to use the 
local IPv4 Address of the routers first interface. IPv4 Addresses seem like a good 
candidate for the Router ID as they are both locally unique 32 bit identifiers. This 
solution would be likely to provide a unique ID that doesn't require configuring by 
the user.
 
However, as we head forward into a world dominated by IPv6, we are likely to see the 
use of IPv4 addresses diminishing. IPv6 addresses are not such a good candidate for 
Router ID for many reasons. Firstly they are not 32 bits, so it would be difficult 
to create an identifier known to be unique from this value. Secondly an interface is 
likely to have many IPv6 Addresses. Of these addresses the only address it is 
guarenteed to have is a link local address, which is not guarenteed to be unique 
across the whole network. 

\subsection{DHCP6-PD}
\todo{Add a description of the actual DHCP6-PD packet} 
Dynamic Host Configuration Protocol for IPv6 - Prefix Delegation, is a method for 
handing a prefix from one router to another. It is then at the disgression of the 
recieving router how it wishes to hand out the prefixes contained in the prefix of 
the PD message. 

One solution to the prefix assignment issue would be to simply use a hierachical 
system. Each router splits up the prefix recieved from an upstream router and 
delegates an equal size prefix to each of the downstream routers connected to it. A 
positive side of this approach is that it is very simple. The negative is that if 
the network is unbalanced, it is possible that one side of the network could end up 
being delegated a far larger address space than it requires, while the other side is 
left without enough address space.  

\subsection{Manual Configuration?}
It would also be possible to just accept the fact that as networks become more 
complex, they will require more configuration. In a current home network, many end 
users don't have any understanding of how the system they have in place works, and 
wouldn't want to touch the configuration anyway. Home users may end up with a better 
experience if a trained engineer were to set up their home network for them. This 
would lead to more jobs, but it is a slightly defeatist attitude, and isn't strictly 
true. 

\subsection{Other Routing Protocols}
Of course any other routing protocol is a candidate for this kind of solution. OSPF 
was chosen as it as good convergence rates and doesn't suffer from many problems due 
to misconfiguration of the network - such as plugging two interfaces into the same 
link.

\section{How OSPF works}
\todo{Show that I understand fully how OSPF works.}
Open Shortest Path First (OSPF) as previously mentioned is a link-state routing 
protocol. A complete image of the network is built up by flooding Link State 
Advertisments (LSA) across the network. 

A variety of different message types exist in OSPF, including Hello messages, sent 
out periodically to discover new adjacencies. Database Descriptions, Link State 
Requests, Link State Updates and Link State Acknowledgments, all of which have 
fairly self descriptive names.

The routers store the LSAs that they know about in their Link State Database (LSDB). 
To identify which router originated an LSA, each router is given a unique Router ID.

\subsection{Links State Advertisments}
There are Several different LSA types in OSPF. Each LSA is originated by a router on 
the network and forwarded on by other routers depending on its flooding scope. Each 
LSA contains some information about the network. 
 
The two most important LSA types are the Router LSA and the Network LSA. Together 
they can be used to build up a complete picture of the areas that the router belongs 
to. 

\subsubsection{Router LSA}
The Router LSA is originated by each router and is flooded to the whole area. Router 
LSAs contain information about the state of the router and its capabilities. They 
also contain information about the interfaces attached to the router, the types of 
these interfaces, and the metrics associated with using these interfaces. 

\subsubsection{Network LSA}
Network LSAs are originated by just one router for each subnet, they are flooded to 
a single area. The router that must originate this LSA is called the Designated 
Router (DR), and is elected by folowing the steps of the Designated router Election 
algorithm. The Backup Designated Router (BDR) is elected in a similar way, and must 
be prepared to start emmiting a Network LSA if the DR becomes unreachable. 

The Network LSA contains a list of routers that are connected to that subnet. There are no weights associated with these connections as they are covered by the Router LSAs. 

\subsubsection{Other LSAs}
There are also several other LSA types that are defined by various different drafts. 
Many of these LSA types are irrelevent in this project as they are concerned with 
networks that consist of many areas, or networks that are concerned with details 
outside of the current Autonomous System (AS). Since OSPFv3 Autoconfig dictates that 
only one area shall be used, these LSA types can be ignored.   

\subsection{Differences between OSPFv2 and OSPFv3}
The main difference is that OSPFv2 is for IPv4 and OSPFv3 is for IPv6. 

For a discussion of how the Shortest Path First calculations are done. See Appendix 
X. \todo{Add some stuff about djirkstras to the appendix}

- Top-$>$Area-$>$Interface

\section{Implementation Candidates}
There are pleanty of implementations of OSPFv3 to chose from. Although there are 
propietry commercial implemenations available, such as the one provided by Cisco 
routers, I have only considered Open Source packages as these are available for me 
to extend.

\subsection{Bird}
Bird is a fairly lightweight suite of routing protocols. It is written in C. 

There are already two implementations of Autoconfiguration OSPFv3. The first was 
written in C as a branch of Bird. The other, written in lua, was created by using an 
extension to bird that allows LSAs to be processed by externternal programs.

\subsection{Quagga}
Quagga is a more heavyweight package, it is a fork of a now defunct GNU project 
called Zebra. Quagga aims to be compatible with a wide variety of platforms, and 
doesn't make much use of external libraries. There are currently no public
implementations of autoconfig OSPFv3.

\subsection{XORP}
XORP is annother heavy implementation. It is written in C++, there aren't any publicly available versions of Autconfig OSPFv3. It seems that the activity surrounding XORP is in decline. 

\section{Chosen Implementation}
I chose to base by project on Quagga. Firstly I found that after subscribing to all 
three developer mailing lists, Quagga seemed to be the most active project. Secondly 
I liked the clear separation between the code for OSPFv2 and OSPFv3 that Quagga 
offered, since my project would only be concerned with IPV6. Finally I felt that 
Quagga was a good choice as there weren't any existing implementations, so my 
contributions would be more useful to the community.

= A hand wavy explanation?

\chapter{Design}
A detailed design

= Discuss How the drafts specify it all to work?

\section{Simplifications}

To allow OSPFv3 to function without requiring any coonfiguration, a number of 
simplifications must be made to how it runs. These simplifications restrict some of 
the fine tuning required by large enterprise networks, but shouldn't severely reduce 
the functionality offered to home users. 

\subsection{Operation on One Area}
As the networks that are being configured are unlikely to be very large, there are 
no real benefits gained from using areas. In autoconfiguration OSPF, all interfaces 
on all routers will be part of Area 0.0.0.0, the backbone area.

Removing mutliple areas also reduces the complexity in other aspects. Various 
summary LSA types are no longer needed, and the flooding scope of area now covers the whole network.

\subsection{Default Values}
The draft specifies that various default values are used. These default values are 
mainly values that were proposed as optional defaults in the original RFCs, and are 
now specifed as obligatory defaults for Autoconfig OSPF. 

Firstly, with the exception of those that clearly should not run OSPF, such as those 
that are manually configured, or those connected directly to an ISP, all interfaces 
should be autoconfigured as the correct type. Most OSPFv3 implementations already do 
this. 

Secondly, a default HelloInterval of 10 seconds is used, and a default 
RouterDeadInterval of 40 seconds is used. There is also an optional reduction of the 
router inactivity time from that of RouterDeadInterval down to a minimum of 
HelloInterval + 1 seconds. 

Finally, with regard to OSPFv3 Address Families, all interfaces should use the 
default value of 0 as their Interface Instance ID. This values specifies that the 
interface uses unicast IPv6.  

\section{New LSA Type}
\todo{Stick a diagram in the Appendix}
The OSPFv3 Autoconfiguration draft specifies a new LSA type to ensure that there are 
no conflicts in Router ID across the network. This new LSA is known as an Auto 
Configuration LSA and currently uses an experimental Type value. The value will 
eventually be assigned by IANA. 

The AC-LSA is designed to be extensible, the aim is to allow other bits of informati
on to be passed around in these packets. 

\subsection{TLVs}
\todo{Confirm that length is in bytes}
The payload of the the AC-LSA is a set of Type-Length-Value (TLV). Each TLV defines 
the type, or meaning, of the data; the length that it occupies in BYTES(?); and then 
the data itself. 

\subsubsection{Router-Hardware-Fingerprint TLV}
The Router-Hardware-Fingerprint (RHWFP) TLV is specified in the Autoconfiguration 
draft. It's value is  an identifier that should be unique for its originating router across the whole network. 

RHWFP TLVs are used to ensure that there are no conflicts in the Router ID across the network.

\subsubsection{Aggregated Prefix TLV}
The Aggregated Prefix TLV is defined in the Prefix Assignment draft, it represents a short IPv6 prefix delegation that can be used to assign prefixes to the different subnets. 

An aggregated prefix should have a length of less than or equal to /64. Typically an ISP is expected to provide a /48.

Aggregated prefixes have many possible sources, the main two being DHCP6-PD and manual configuration. 

\subsubsection{Assigned Prefix TLV}
The Assigned Prefix TLV is also defined int the Prefix Assignment draft. A router adds an Assigned Prefix TLV to its AC-LSA when it makes an assignment from an Aggrgegated prefix. 

Each assignment is made to just one interface on the network, conflicts are resolved if they occur, and is exactly /64 as this is the length used by SLAAC.

\chapter{Implementation}
The implementation

+ How I actually did the dirty

\section{Core Algorithms}
\todo{This stuff might fit better into the previous section}
There are many algorithms used in autconfig OSPFv3 to ensure that the network is 
configured correctly.

\subsection{Router Hardware Fingerprint Generation}
The Router Hardware Fingerprint is a value that is based on some properties of the 
router that has a high probability of being unique on the network. 

In my implementation I chose to use a sum of the MAC addresses of the attached 
interfaces. Although, the MAC addresses are not guarenteed to be unique as 
manafacturers could cut corners - if they are not, there are likely to be bigger 
problems anyway.

\subsection{Router ID Generation}
According to autoconfig OSPF, the Router ID should be a pseduo-random number,
based on the Router Hardware Fingerprint. As the router will need to reassign
its Router ID to a new number if there is a conflict, a seed value must be
stored. This value is passed rand\_r function call, which modifies it to
maintain the state of the random sequence.

I found one of the harder parts of the project was cleanly shutting down OSPF,
changing it's router ID, and then bringing it back up again. 

\subsection{Router ID Conflict Resolution}
As the Router IDs are assigned randomly, there is a chance that there could be a 
collision, where two routers on the network assign themselves the same Router ID. 
This means that a mechanism for resolving this conflicts is needed. 

The drafts specifies two separate mechanisms for detecting, and resolving, 
conflicts. 

\subsubsection{Local Collisions}
A local collision, one where there is an adjacency between the conflicting routers, 
is detected when any valid LSA is detected to have the same Router ID but a 
different Link-Local IPv6 address. To resolve this conflict, the router with the 
numerically lower Link-Local Address must generate a new Router ID. 

\subsubsection{Network-Wide Collisions}
To detect collisions that occur for routers that are not directly connected, all 
apparent Self Originated AC-LSAs are inspected to ensure that they are truely Self 
originated LSAs. This is done by checking that the Router Hardware Fingerprints 
match. If they don't then the router with the lower Router Hardware Fingerprint must 
change it's Router ID. 

\subsection{Prefix Assignment}
When a router notices a change in the current set of AC-LSAs in the LSDB, the prefix 
assignment algorithm is scheduled to run. 

During the prefix assignment algorithm, all aggregated prefix - interface pairs are 
examined. If there is not already an assignment, and the router has the highest 
Router ID of all active neighbours on that link, then it must make a prefix 
assignment from this aggregated prefix. 

There is no prescribed method in the draft for making an assignment in the draft, 
beyond the fact that the prefix assigned must not already be an assignment. My 
implemnetation has been written so it is easy to change the method used for picking 
a new prefix for the assignment. 

The current method I currently use is to simply step through the prefixes in 
numerical order, this method has the disadvantage of being likely to collide with a 
prefix alocated by another router in the network concurrently allocating prefixes. 
However this can be an advantage during the testing phase as it makes these cases 
more common, and as such handling these cases will undergo greater testing. The 
other advantage is that this algorithm is guarenteed to either halt, or produce an 
error once the address space has been exhausted. 

Another method, used by one of the bird implementations is to allocate at random for a fixed number of tries, then if that failed to fall back to sequential allocation. 

\todo{This may well be the place to start talking about RADVD}

\subsection{Prefix Conflict Resolution}
\todo{Review the code to see how many places do this}
Prefix conflicts are dealt with at many different levels in the algorithm. 

\subsection{ULA Generation}
If a there are no aggregated prefixes in the routing domain after a specified amount 
of time (120 second?) then the router with the highest Router ID must generate a 
Unique Local Address (ULA) to allow local connectivity. ULAs are similar in concept 
to an IPv4 Private address (e.g. 192.168.1.0) although are generally not used with 
NAT. 

The ULA generation algorithm is defined in [AN RFC], and aims to be as unique as 
possible. The the ULA is based on a portion of a combination of the current network 
time (in Network Time Protocol (NTP) format), and an interface's 64bit MAC address. 

Once a ULA has been generated it is advertised as though it were a normal aggregated 
prefix. If a router with a higher Router ID becomes reachable, or an aggregated prefix from another source is recieved, the ULA is deprecated. 

\section{Quagga}
Qaugga is written in C89, following the GNU coding standards. These rules help to 
improve the portability of the code, meaning that if there is a C compiler, and the 
operating system is POSIX compiliant, then Quagga is likely to compile for it. 
Quagga doesn't support Microsoft Windows (or WinDoze as the GNU coding standards 
would put it). 

At first I found some of the stylistic guidelines of the GNU coding standards
unusual, and difficult to read. Placing a space between function calls and the 
opening parenthesis of their parameter list, declaring the return type of a function 
on a separate line to the name of the function, and placing the opening curly brace 
of a block on a new line all seemed strange to me at first. I am ashamed to admit 
that when I first opened one of Quagga's source files, it took me a long hard stare 
to realise that I was looking at a function declaration. Fortunately as time went
on, reading, and writing, this formatting became second nature to me. 

Quagga tends to avoid relying on any external libraries. There is a library folder in the repository.

*Talk about the style of Quagga's code?
==$>$ Do I put in about Threads here?
Cite the Hacking Guide and the GNU coding standards.

- The issues that I found
    Little Endian vs Big Endian
    
- Something about RADVD
    
-Chat about C?

= I would like to have something about C

\begin{itemize}
  \item Structs
  \item MEMORY LEAKS?
  \item Lack of Objects - How do you write clean code without objects
\end{itemize}

\chapter{Testing}
Testing strategy and results

\section{Unit Testing}
Created unit tests to give me confidence that the code I wrote was correct. 

As I was using C I was unable to make use of a unit testing framework such aas 
JUnit. I also felt basing my code on external libraries tended to go against the 
ethos of the Quagga.

\section{Netkit} 
Made testing fresh and easy. 

REALLY LET TIM KNOW HOW GOOD IT IS!

Stick a screen shot in the appendix :)

\section{Tunnel Broker}
Should I talk about how I configured it all to work with a tunnel broker here?

Using Hurricane Electrics Tunnel broker service, I was able to give my network
truely global connectivity

\section{Alix2d3}
Bit harder, but still did it anyway.

\section{Interoperability}

Tested against other existing implementations.

Assertions -$>$ IF I actually put some in my code\ldots

\chapter{Project Management}
/* Might be able to sneak this bit into the Appendix */

\section{Gantt Charts}
The most exciting part of this document!

\section{Tools}
With a project like this, there must be lots of tools used. 

\subsection{Git}
You need a version control system!

GitHub is really good. And I'm pretty sure they put a lot of effort into backup.

\subsection{Trello}
Gotta love that task management system

\section{Risk Assesment}
Asses that risk; Like a Boss

\chapter{Evaluation}
A critical evaluation

\section{What Went Well?}

\begin{itemize}
\item Wrote loads of C
\item Ended up with something that seems to work.
\item Somewhere round here talk about how Distributed algorithms require a different type of though to understand. They often seem far too little work has been done locally...
\end{itemize}

\section{What Could Have Gone Better?}

\begin{itemize}
\item Not much really :p
\item Could have spoken more to Jarri
\item More feedback from community??
\end{itemize}

\chapter{Conclusions And Future Work}
Conclusions and future work

\section{Conclusions}

It is possible to do

Some things aren't very prescriptive in the draft

\begin{itemize}
\item It works! 
\item Release source code to community
\item Code reviews
\item Attend IETF 87
\end{itemize}

\section{Future Work}
Some things I have thought about, but haven't actually done yet.

\section{DHCP6-PD}
Would be nice to have quagga intercept DHCP6-PD messages and use the prefixes as aggregated prefixes. 

At the moment I didn't manage to do this and I it was deemed beyond project scope.

\section{Source Based Routing}
I might actually create a branch of my code to do a quick proof of concept. At the moment, 
it would be beyond the scope of the project to implement this. 

\section{Multi-hop Service discovery}
I think that this would be worthy of a third year project on it's own. 
Too little standardization - Really something apple would need to contribute to. 

== I DO WANT TO HAVE A GO AT DOING SERVICE DISCOVERY.
-- There is a possibility I will be able to get an RFC out of it :D :D :D

\section{Security Considerations}
Currently the only security that exists in the implementation is IPSec (IP level security). It might be neccessary for future implementations to implement security.

Security was a feature of OSPFv2 but it was removed in OSPFv3 in favour of using 
IPSec. In some circumstances, IPSec is considered difficult to employ. RFC6506 
reintroduces OSPF level security. 

This RFC could be implemented to add Security to Autoconfigured networks. It would 
be possible to provide CPE that comes with a randomly generated key similar to the 
randomly generated Wireless PSKs that currently come with wireless access points.

\section{Realization}
It would be nice to see this kind of system deployed in the home. 
Requires ISPs to start using IPv6.
Outside my control. 
Note: Andrews and Arnold.

=$>$ Things beyond my controls - IPv6 etc

\pagebreak

\printnomenclature

\addcontentsline{toc}{chapter}{Nomenclature}
\pagebreak

\bibliography{FinalReport}{}
\bibliographystyle{unsrt}
\addcontentsline{toc}{chapter}{Bibliography}

\appendix 
%\appendixpage

\chapter{Appendices}
Appendices providing detailed technical references

\end{document}
