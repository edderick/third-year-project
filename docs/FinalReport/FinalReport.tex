\documentclass[12pt,a4paper,twoside]{report}
\usepackage[utf8]{inputenc}
\usepackage{amsmath}
\usepackage{amsfonts}
\usepackage{amssymb}
\author{Edward Seabrook} 
\title{Third Year Project Final Report}

\usepackage{nomencl}
\makenomenclature

% Makes Chapter heading look like Section heading
\usepackage{titlesec}
\titleformat{\chapter}% reformat chapter headings
    [hang]% like section, with number on same line
    {\Large\bfseries}% formatting applied to whole
    {\thechapter}% Chapter number
    {0.5em}% space between # and title
    {}% formatting applied just to title

% Save a bit of space by giving all headings less room
\titlespacing*{\chapter}{0pt}{0pt}{0pt}
\titlespacing*{\section}{0pt}{0pt}{5pt}
\titlespacing*{\subsection}{0pt}{0pt}{5pt}
\titlespacing*{\subsubsection}{0pt}{0pt}{5pt}
\titlespacing*{\paragraph}{0pt}{0pt}{5pt}

% Set margins to required
\usepackage[top=2.4cm, bottom=2.4cm, left=2.95cm, right=2.95cm]{geometry} 

% Sort out margins for todonotes
\setlength{\marginparwidth}{3cm}
\reversemarginpar

% Set paragraph spacing to required
\setlength\parindent{0pt}
\usepackage[parfill]{parskip}

\usepackage{hyperref}
\usepackage{todonotes}
\usepackage{listings}
\usepackage{appendix}
\usepackage{pdflscape}
\usepackage{cite}
\usepackage{wrapfig}

\hypersetup{colorlinks=false, pdfborder={0 0 0}, }

% Not totally sure
\usepackage{fancyhdr} 
\pagestyle{fancy} 
\renewcommand{\headrulewidth}{0pt} 
\lhead{}\chead{}\rhead{}
\lfoot{}\cfoot{\thepage}\rfoot{}

% Number and show in ToC to a deeper level
\setcounter{secnumdepth}{3}
\setcounter{tocdepth}{3}

\def\nomlabel#1{\textbf{#1}\hfil}

\begin{document}

% Include title page

\begin{titlepage}

\begin{center}


% Upper part of the page
%\includegraphics[width=0.15\textwidth]{./logo}\\[1cm]    

\LARGE Electronics and Computer Science\\
Faculty of Physical and Applied Sciences\\
University of Southampton
\\[1.5cm]

\href{mailto:ejfs1g10@ecs.soton.ac.uk}{Edward JF Seabrook}\\[0.5cm]

\today \\[1cm]
{\bfseries A Tool to Simplify Network Administration in the Modern Home}\\[1.5cm]

\vfill

% Author and supervisor
\large
Project Supervisor: 
Dr.~T \textsc{Chown}\\

\large
Secondary Examiner:
Dr.~KP \textsc{Zauner} 

\vfill

A Project Progress Report Submitted for the Award of Computer Science

\end{center}

\end{titlepage}


\begin{abstract}
As home networks become more and more complex, it is inevitable that they will
require splitting into multiple subnets. As the average home user is unable to
configure a router, a minimal configuration solution is required. In this
project I produced an implementation of the Open Shortest Path First for IPv6
(OSPFv3) routing protocol, that requires no configurations to run in the home.
The implementation is based mainly on two drafts published recently by the
IETF\@\nomenclature{IETF}{Internet Engineering Task Force}. 
\end{abstract}

\tableofcontents
\clearpage

\chapter*{Acknowledgements}
\todo{Write as prose?}
I would like to thank the following people for all their help, support and
influence during this project:
\begin{itemize}
\item Tim Chown -- Offered support as project supervisor
\item David Lamparter -- Responded to my queries about how Quagga works
\item Jari Arkko -- Wrote the drafts that I was implementing
\item Markus Stenberg \& Benjamin Paterson --- Provided implementations against which 
      to interop test
\item Friends \& Family -- Proof read this document.
\end{itemize}

\chapter{Introduction}
The title of my project is ``Implementing Zero Configuration OSPFv3''. During
the project, I have modified an open source implementation of the OSPFv3
routing protocol to comply with two recently published, and very topical,
internet drafts. The changes allow routed networks to be set up without the
need for manual configuration.

\section{The Problem}
As computing becomes more and more ubiquitous the complexity of an average home
network will increase. Not only will the number of devices rise, but with the
advent of sensor networks and home automation, the quantity of speeds,
types and topologies present will grow. To ensure these more complex networks
run smoothly, they should be separated into multiple subnets. Routing is
required to allow communication between these subnets. 

Unfortunately, the vast majority of home networks do not have a professional
System Administrator around to configure and maintain them. This means a
network should require no manual configuration to work.

Another issue that is addressed in this project is IPv4
\nomenclature{IPv4}{Internet Protocol Version 4} address space exhaustion. IPv6
is the main candidate for replacing IPv4, and with its much larger address
space solved the exhaustion problem. Home users do not want to configure IPv6
addresses by hand, so a mechanism is needed for assigning prefixes to the
networks in the home. 

\section{The Goals}
This project set out to meet many goals, over the course of the project, as my
understanding of the problem improved, the goals have transformed slightly.

\subsection{Project Goals}
The goals for the project are:

\begin{itemize}
\item Allow multiple subnets in the home; allow hosts to communicate from one
  subnet to another.
\item Distribute IPv6 subnet prefixes across a home network in an efficient
	manner. 
\item Require no configuration: The user should not need to enter any settings,
  they should simply plug in the router, and it should work.
\item Produce something of value for the community. The code I produce should
  adhere to style and quality guidelines to ensure it is useful.
\item Verify past implementations of the drafts.
\item Discover any ambiguities in the drafts.
\end{itemize}

\subsection{Personal Goals}
I also had a set of personal goals that I aimed to achieve over the course of
the project:
\begin{itemize}
	\item Develop a good understanding of the OSPF routing protocol.
	\item Improve my understanding of the process of the creation of internet
		standards.
	\item Obtain the ability to implement network protocols.
	\item Increase my programming ability (Mainly C or C++).
	\item Learn about new tools and techniques.
\end{itemize}

\chapter{Background}
To enable me to explain my project in detail, I first need to explain the
concepts that it is based on.

\section{Home Networks}

\begin{figure}
\label{typical_homenet}
\begin{center}
  \includegraphics[width=\linewidth]{../Diagrams/Network/TypicalHomenet.png}
	\caption{The topology of a typical present day homenet.}
\end{center}
\end{figure}

Modern home and small/home office (SOHO) \nomenclature{SOHO}{Small Office Home
Office} networks tend to follow the same pattern. A single ISP provides a
connection to the internet using the phone line or television cable.  The ISP
will typically provide a single IPv4 address (/32)\footnote{This is CIDR
notation, it represents the number of bits used to identify the network the
host.}. Usually this is an unstable (dynamic) allocation -- a static IP is
usually only offered as a premium service. 

To allow multiple hosts on the network, Network Address Translation (NAT)
\nomenclature{NAT}{Network Address Translation} is usually employed. With NAT,
the router rewrites the address of the packets from the global IP address to a
local IP address. The address range 192.168.1.0/24 is often used for these
local IP addresses. Ports number are often used to multiplex between the
different hosts and applications on the network. To allow programs like Skype
to function, Universal Plug and Play (UPnP) \nomenclature{UPnP}{Universal Plug
and Play}, a technology that allow automatic port mapping on a network, is
frequently deployed. Local IP addresses are assigned to the hosts by the router
using the Dynamic Host Configuration Protocol (DHCP)\nomenclature{DHCP}{Dynamic
Host Configuration Protocol}. 

A single item of Customer Premises Equipment (CPE) \nomenclature{CPE}{Customer
Premises Equipment} is usually supplied to the customer. These devices are
referred to by many names\footnote{Names include Super Hub, HomeHub and
WirelessBox}, commonly as a ``Router''.  This device provides the modem for
connecting to the internet, an Ethernet switch with around four ports, and a
wireless access point (802.11) that tends to be bridged to the wired network to
provide just one subnet. A diagram of a typical home network can be seen in
Figure~\ref{typical_homenet}.

\section{Campus and Enterprise Networks}
\begin{figure}
\label{corporate_net}
\begin{center}
	\includegraphics[width=\linewidth]{../Diagrams/Network/CorporateNetwork.png}
	\caption{The topology of a typical corporate network.}
\end{center}
\end{figure}
Larger networks like the ones found on University campuses and larger
businesses tend to employ many of the more advanced features of networking.
They are often split into multiple subnets separating the physical locations,
work groups and expected use of the hosts. One potential corporate network
topology can be seen in Figure~\ref{corporate_net}.

\todo{Is there any evidence to support this?}
In the early days of the internet, many organisations were able to obtain large
chunks of IPv4 address space -- MIT famously has the control of a /8 address
range. The University of Southampton was able to obtain a /16 (152.78.0.0)
giving them around 65000 addresses to play with. Having this many addresses
allows each machine on the network to be globally addressable -- at first this
may seem to be insecure, but a good firewall is far more effective protection
than NAT\@. Even with this many addresses, the University still struggles. This
is due at part to address space wastage, the now deprecated ISS wireless
service still takes up a large proportion of the address space.

Although it is not yet common place, the networks run by ECS are dual stack,
the machines are allocated both IPv4 and IPv6 addresses. The 6th June 2012 was
the World IPv6 Launch, on this day many websites, including Facebook and
Google, began offering IPv6 connectivity to their users. This demonstrates than
IPv6 is mature enough to be deployed in large networks.

It should be noted however, to run a network the size of the University's,
based on todays technology, many specialist staff are required to configure the
network and ensure its day-to-day operation. This is something that the vast
majority of households are not able to obtain.

\section{Future Networks}
\begin{figure}
\label{future_net}
\begin{center}
	\includegraphics[width=\linewidth]{../Diagrams/Network/FutureHomenet.png}
	\caption{The topology of a hypothetical home network.}
\end{center}
\end{figure}
In the future users will expect more from their home networks. We are likely to
see a dramatic increase in the number of devices in the average home, ranging
from more laptops, smart phones and tablets, to more sensors (e.g.\@
temperature/humidity)  and home automation devices. A hypothetical home network
of the future is depicted in Figure~\ref{future_net}.

A home network with this quantity and heterogeneity of traffic is likely to
experience slow downs and other issues. These problems arise for two reasons:
Firstly sensor networks run very slowly, where as Gigabit Ethernet networks run
very quickly -- this difference in speed can result in both networks running
even slower. The other issue is the vast amounts of multicast and broadcast
traffic that will build up on networks with many devices performing functions
such as service discovery. It would be beneficial to split the network up into
multiple subnets to ensure the best performance from each network. 

Guest networks are already an option provided by some high end wireless access
points. By using Virtual Access Points (VAPs) \nomenclature{VAP}{Virtual Access
Point} guest clients can connect to the internet, but are restricted from
having full access to the local network. To facilitate this, hard coded local
IPv4 addresses are usually used, with the advent of IPv6 and globally valid
addresses, a mechanism to assign IP addresses and route traffic is required. 

The Internet Engineering Task Force's (IETF) Homenet working group
\cite{homenet} are working on the architecture of the home network of the
future. Their work mainly focuses on addressing, routing, topology and security
in home networks. 

\section{IPv6}
To communicate with a computer, a way of uniquely identifying it is required.
Currently, IPv4 is the dominant technology for addressing computers on the
internet. When IPv4 was designed, it was not foreseeable that we 4 billion
addresses would be needed, so a neat 32 bit identifier was used.  In 2011 the
Internet Assigned Numbers Authority (IANA) \nomenclature{IANA}{Internet
Assigned Numbers Authority} ran out of IPv4 addresses \cite{potaroo}. 

One solution to the IPv4 address space exhaustion problem is migration to
IPv6; a protocol that is fundamentally similar to IPV4, but instead uses 128
bit addresses. This gives us far more addresses to work with -- in fact a total
of $3.4028237\times10^{38}$ unique addresses. With estimates for the total
number of grains of sand on Earth being around $7.5\times10^{18}$, the number
of stars visible from Earth falling at about $7.0\times10^{22}$, and the
estimated human population of Earth being a mere $7.0\times10^{9}$,
this address space should be sufficient for at least the foreseeable future. 

IPv6 address ranges are given in Classless Inter-Domain Routing (CIDR)
\nomenclature{CIDR}{Classless Inter-Domain Routing} notation. In this notation
an IPv6 address is followed by a slash and a number between 0 and 128 that
represents the number of bits in the IPv6 address that identify the network --
the remaining bits identify the host. CIDR notation replaces subnet masks used
in IPv4.

IPv6 prefixes tend to be /64, as this is required by Stateless Address
Autoconfiguration (SLAAC)\nomenclature{SLAAC}{Stateless Address
Autoconfiguration}. A /48 prefix may be allocated by an ISP to allow the user
to assign multiple subnets on their network.

\section{Routing}
For packets of data to find their way between subnets, they require routing.
The routing we are concerned with is Layer 3 (Network/IP Layer)
routing\footnote{For a brief explaination of OSI layers see
Appendix~\ref{osi}}. When an IP packet is sent, its address is compared with
those stored in the kernel's routing table, and then transmitted using the
appropriate interface. Networks contain many computers, known as routers, whose
sole purpose is to forward network traffic.

A number of different protocols exist to build up these routing tables:

\subsection{Interior gateway protocols}
Interior Gateway Protocols (IGPs) \nomenclature{IGP}{Interior Gateway
Protocols} are concerned with building the routing tables for a single
autonomous system (AS) \nomenclature{AS}{Autonomous System}. ASs are generally
managed by ISPs or other very large organisations.  

\subsubsection{Distance Vector}
Distance vector routing works by having each router advertise to its neighbors
(directly connected routers) the shortest path it knows about to each subnet.
Although less complex than link-state routing protocols, they are often slower
to converge, and experience what is know as the ``count to infinity'' problem
whereby loops in the network topology can lead to weights increasing to
infinity when a router goes down. 

The most common Distance Vector protocol is Routing Information Protocol (RIP)
\nomenclature{RIP}{Routing Information Protocol}, a routing protocol famous for
defining 15 as infinity. The most recent revision is RIP Next Generation
(RIPng) which offers IPv6 support. IGRP and EIGRP are Cisco's own propriety
alternatives.

\subsubsection{Link-State}
Link state routing protocols work by allowing each router on the network to get
a full understanding of the network topology. From this representation of the
networks topology, the router then performs some calculations to work out the
best routes. The two most common Link-State routing protocols, Open Shortest
Path First (OSPF) \nomenclature{OSPF}{Open Shortest Path First} and
Intermediate System to Intermediate System (IS-IS)
\nomenclature{IS-IS}{Intermediate System - Intermediate System}, both use
Dijkstra's algorithm to calculate these shortest paths. 

OSPF and IS-IS are both very similar protocols, the main difference is that
IS-IS as a Layer 2 protocol is Layer 3 agnostic, whereas OSPF is dependent on
IPv4 or IPv6 depending on the version.

\subsection{Exterior Gateway Protocols}
Exterior Gateway Protocols (EGP) \nomenclature{EGP}{Exterior Gateway Protocol}
are used to establish peerings between ASs. EGPs typically run only on edge
routers and construct the route tables based on policy rather than the
technical distance of the destinations. In this project I was not concerned
with this kind of routing protocol.  The most widely used EGP today is Border
Gateway Protocol (BGP) \nomenclature{BGP}{Border Gateway Protocol}.

\chapter{Literature Review} 
The project was based on two closely related internet drafts, they were both
published very recently, and have the same authors. If these drafts are
approved by the IETF they will end up becoming Request for Comments
(RFC)\nomenclature{RFC}{Request for Comments}, the technical documents that
specify how the internet works. As RFCs these documents will be placed on the
standards track and will be labeled as ``Proposed Standards''. If they are
successful in meeting certain criteria, they may eventually become ``Internet
Standards''\cite{rfc6410}.

\section{OSPFv3 Autoconfiguration}
This draft\cite{draft-ietf-ospf-ospfv3-autoconfig-02} describes how OSPF for
IPv6 can be extended to run without requiring configuration. The draft begins
by enforcing some default values, and restricting a few features -- such as
allowing only the backbone area to be used. It then specifies a mechanism for
generating Router IDs and ensuring that there are no conflicts across the
network. This is achieved by the introduction of a new Link State Advertisement
(LSA) \nomenclature{LSA}{Link State Advertisement} type -- the Autoconfiguration LSA.. 

The most recent release of this document was on 15th April 2013. Although this
was towards the end of my project, the changes were only minor and I was able
to incorporate them into my implementation.

\section{OSPFv3 Prefix Assignment}
Building on the work of the Autoconfiguration draft, this
draft\cite{draft-arkko-homenet-prefix-assignment-03} solves the issue of
delegating IPv6 network prefixes from a larger pool of IPv6 addresses. As well
as offering a realistic solution to a problem, the draft also demonstrates the
extensibility of the Autoconfiguration draft. 

A method for breaking up a short prefix delegation into many network prefixes
is given, along with a strategy for avoiding conflicts across the network.
Typically an ISP will allocate a customer a /48 IPv6 prefix to cover their
whole home, each network in their home then requires a /64 IPv6 prefix.

An update to this document was published on 23rd October 2013, it expired on
26th April 2013, although it is likely a refresh will be released soon.

\section{Other Important Drafts}
There are several other documents that are relevant to this project. These are
both articles that are relevant as background knowledge and papers that have
been referenced from one another.

\subsection{Home Networking Architecture for IPv6} 
This paper\cite{draft-ietf-homenet-arch-07} puts forward a vision of future
home networking. An architecture for future home networks is proposed, along
with a discussion of emerging technologies. 

\subsection{OSPF Version 2}
I studied RFC2328\cite{rfc2328} in depth to understand the OSPF Protocol. The
memo is surprisingly easy to read, and offers a complete discussion of the data
structures and algorithms the OSPF protocol employs to build routing tables for
IPv4. 

Although the document contains many details that are irrelevant to this project, 
including a long discussion on areas, and details of the SPF algorithms, I found
the understanding I gained through studying it invaluable to my project.

\subsection{OSPF for IPv6}
RFC5340 defines OSPFv3\cite{rfc5340}; it builds on RFC2328, to give the
reader a clear understanding of OSPFv3 by explaining the differences between
from OSPFv2. 

\subsection{Zeroconfiguration OSPF}
This document\cite{draft-dimitri-zospf-00} is an old draft that had similar
aims to the OSPFv3 Autoconfiguration draft. This draft focuses on allowing
autoconfiguration for both IPv4 and IPv6, this made it a more complex and
restrictive solution. 

\subsection{The OSPF Opaque LSA Option}
RFC5250\cite{rfc5250} introduces the opaque LSA\@. An opaque LSA is a generic
LSA type that can be used to transmit application specific data. The concept is
similar to that used in the autoconfiguration draft, but it does not make use
of Type-Length-Values (TLVs)\nomenclature{TLV}{Type-Length-Value}. I found the
discussion about the reachability of LSAs useful.

\subsection{Additional Reading}
While learning about this area of networking, there were many other documents
that were either referenced from the more relevant drafts or defined a concept
that had been mentioned. This list includes:

\begin{itemize} 
  \item {\bf RFC 5838} Support of Address Families in OSPFv3 \cite{rfc5838}
	\\ Extensions to allow different kinds of address to be used with OSPFv3.
  
\item {\bf RFC 6506} Supporting Authentication Trailer for OSPFv3  \cite{rfc6506}
	\\ Adds a level of security back into OSPFv3.

	\item {\bf RFC 3630} (Updated by 4230 \& 5786) Traffic Engineering Extensions
	for OSPF \cite{rfc3630} \cite{rfc4230} \cite{rfc5786} 
	\\ Extra set of LSAs to pass information useful for traffic shaping etc.\@
  
\item {\bf RFC 4862} IPv6 Stateless Address Autoconfiguration (SLAAC) \cite{rfc4862}
	\\ Explains how an IPv6 networks can be established using NDP.\@

\end{itemize}

\section{Books}
There were a few books that were relevant to this project:
\begin{itemize}
  \item OSPF: Anatomy of an Internet Routing Protocol \cite{OSPFAIRP}
  \item OSPF: Complete Implementation \cite{OSPFCI}
  \item OSPF and IS-IS \cite{OSPFvsISIS}
\end{itemize}

These books were useful for clarifying things that were not clear from the
RFCs. They were also useful for reinforcing my general understanding of the
protocol, as they gave good overviews, written from different angles. 

Unfortunately the books did not cover OSPFv3 extensively because it is a
relatively recent development. There was no mention at all of more recent
drafts since they are very much on the cutting edge.

\chapter{Analysis}
\section{Alternatives}
There are also other approaches that can be used to solved the problems
addressed by these drafts. The different solutions all have their merits -- to
get a real understanding of the problems with an approach, it is necessary to
attempt to implement it.

\subsection{IPv4 Address as Router ID}
One possible solution to the Router ID assignment problem would be simply to use
the local IPv4 Address of the routers first interface. IPv4 Addresses seem like
a good candidate for the Router ID as they are both locally unique 32 bit
identifiers. This solution would be likely to provide a unique ID that does not
require configuring by the user.
 
However, as we enter a world dominated by IPv6, the use of IPv4 addresses will
diminish. IPv6 addresses are a poor candidate for Router IDs for many reasons.
Firstly as they are larger than 32 bits, it would be difficult to create an identifier
certain to be unique from them. Secondly an interface is likely to have
many IPv6 Addresses. Of these addresses the only address it is guaranteed to
have is a link-local address, which does not have to be unique across the
network. 

\subsection{DHCPv6-PD}
\begin{figure}
\label{pd-option}
\begin{center}
	\includegraphics[width=\linewidth]{../Diagrams/Packets/pd-option.png}
	\caption{The DHCPv6-PD option.}
\end{center}
\end{figure}

DHCP for IPv6 Prefix Delegation (DHCvP6-PD)\nomenclature{DHCPv6-PD}{DHCP for
IPv6 Prefix Delegation}, is a method for handing a prefix from one router to
another. The receiving router decides how it wishes to hand out the prefixes
contained in the prefix of the PD message (Shown in Figure~\ref{pd-option}). 

One solution to the prefix assignment issue would be to simply use a
hierarchical system. Each router splits the prefix received from an upstream
router and delegates an equal size prefix to each downstream router. A positive
side of this approach is that it is very simple. The negative is that if the
network is unbalanced, one side of the network would be delegated a far larger
address space than it requires, while the other side is starved of prefixes.
Without some extension there is no way of communicating address space
requirements to upstream routers.   

\subsection{Manual Configuration}
Another option is to accept that as networks become more complex, they will
require more configuration. In a current home network, most end users do not
understand how their network works, and have no desire to touch the
configuration anyway. Home users may end up with a better experience if a
trained engineer was sent by their ISP to set up their home network for them.
This would lead to more jobs, but would be inconvenient and expensive for end
users.

\subsection{Other Routing Protocols}
Other routing protocols make good candidates for this kind of solution. OSPF
was chosen as it as an IETF standard with good convergence rates and does not
suffer from many problems due to misconfiguration of the network -- such as
plugging two interfaces into the same link. No matter what routing protocol is
chosen as the eventual standard in home networks, this project will certainly
be an influence to its design.

\section{How OSPF works}
Open Shortest Path First (OSPF) is a link-state routing protocol. An image of
the whole network is built up by flooding Link State Advertisements (LSA)
across the network. Dijkstra's algorithm is used to build up routing table from
the LSAs.

Several different message types exist in OSPF\@. These include:
\begin{itemize}
 \item Hello -- Sent out periodically to discover new adjacencies.
 \item Database Descriptions -- Which lists the LSAs in the routers LSDB (see below).
 \item Link State Requests -- Used to ask a router to send an LSA.
 \item Link State Updates -- Used to send an LSA.
 \item Link State Acknowledgments -- Sent on the receipt of an LSU.
\end{itemize}

The routers store the LSAs that they know about in their Link State Database
(LSDB)\nomenclature{LSDB}{Link State Database}. To identify which router
originated an LSA, each router is given a unique Router ID.

\subsection{Links State Advertisements}
\begin{figure}
\label{LSA-header}
\begin{center}
	\includegraphics[width=\linewidth]{../Diagrams/Packets/LSA-header.png}
	\caption{OSPFv3's LSA Header.}
\end{center}
\end{figure}

There are many different LSA types in OSPF\@. Each LSA is originated by
a router on the network and forwarded on by other routers depending on its
flooding scope. Each LSA contains some information about the network. 
 
The two most important LSA types are the Router LSA and the Network LSA\@.
Together they can be used to build up a complete picture of the areas that the
router belongs to. Figure~\ref{LSA-header} shows the structure of
an LSA header. 

\subsubsection{Router LSA}
The Router LSA is originated by each router and is flooded across its own area.
Router LSAs contain information about the state of the router and its
capabilities. They also contain information about the interfaces attached to
the router, the types of these interfaces, and the metrics associated with
using these interfaces. 

\subsubsection{Network LSA}
Network LSAs are originated for each subnet, they are also flooded to its own
area.  The only router that should originate this LSA is the Designated Router
(DR) \nomenclature{DR}{Designated Router}, and is elected with the DR Election
algorithm. The Backup Designated Router (BDR) \nomenclature{BDR}{Backup
Designated Router} is elected in a similar way, and must be prepared to start
emitting a Network LSA if the DR becomes unreachable. 

The Network LSA contains a list of routers that are connected to that subnet.
There are no weights associated with these connections as they are covered by
the Router LSAs. 

\subsubsection{Other LSAs}
There are also other LSA types that are defined.  Many of these LSA types are
irrelevant in this project as they are concerned with networks that consist of
many areas, or interactions that occur outside of the current Autonomous System
(AS). Since OSPFv3 Autoconfig dictates that only one area shall be used, these
LSA types can be ignored.   

\subsection{Differences between OSPFv2 and OSPFv3}
\todo{This looks to be incomplete}
The main difference is that OSPFv2 is for IPv4 and OSPFv3 is for IPv6. 

\section{Implementation Candidates}
Below is a list of Open source implementations of OSPFv3\@. There are also
commercial implementations\footnote{Most notably the one provided by Cisco with
their routers.}, these have not been considered as they are closed source. 

\subsection{Bird}
Bird\cite{BirdHome} is a lightweight suite of routing protocols, written
in C. 

Bird has two implementations of Autoconfiguration OSPFv3. The first was written
in C as a separate branch. The other, extended bird to allow LSAs to be
processed by external programs, and then used Lua to implement the draft.

\subsection{Quagga}
Quagga\cite{QuaggaHome} is a more heavyweight package, it is a fork of now
defunct GNU project Zebra. Quagga is written in C, and aims to be
compatible with many platforms so does not rely on many external libraries.
There are no public implementations of autoconfig OSPFv3.

\subsection{XORP}
XORP\cite{XORPHome} is another heavy implementation. It is written in C++,
there are not any public Autconfig OSPFv3 implementations. The activity
surrounding XORP is low. 

\section{Chosen Implementation}
I chose to base by project on Quagga. I found that after subscribing to all
three developer mailing lists, Quagga seemed to be the most active project.  I
also liked the clear separation between OSPFv2 and OSPFv3 that Quagga offered.
Finally Quagga is a good choice as there were no existing implementations, so
my contributions would be more useful to the community.

\chapter{Specification}
\todo{This section is up for eviction}
To help direct my projects implementation and allow a thorough evaluation, I
came up with the following specification based on the drafts I am implementing.

\section{Functional Requirements}
\subsection{Necessary}
\begin{itemize}
    \item Should allow routing daemons to run without manual configuration
    \item Should efficiently distribute IPv6 Prefixes across the network
\end{itemize}
\subsection{Optional}
\begin{itemize}
    \item Should be interoperable with existing implementations of the drafts
    \item Should generate a ULA if one is required
\end{itemize}

\subsection{Non Functional Requirements}
\subsection{Necessary}
\begin{itemize}
    \item 
\end{itemize}
\subsection{Optional}
\begin{itemize}
    \item 
\end{itemize}

\chapter{Design}

\section{Simplifications}
To allow OSPFv3 to function without configuration, a number of simplifications
must be made to how it runs. These simplifications restrict some of the fine
tuning required by large enterprise networks, but should not remove functions
exploited by home users. 

\subsection{Operation on One Area}
As the networks that are being configured are unlikely to be very large, there
is no benefit gained by using areas. In Autoconfig OSPFv3, all
interfaces on all routers will be part of Area 0.0.0.0, the backbone area.

Removing multiple areas also reduces the complexity in other aspects. Various
summary LSA types are no longer needed, and the flooding scope of area now
covers the whole network.

\subsection{Default Values}
The draft specifies the use of default values. These are mainly values that were
optional in the original RFCs, and are now obligatory defaults for Autoconfig
OSPFv3\@. 

Firstly, with the exception of those that clearly should not run OSPF, (e.g.\@
manually configured, or connected directly to an ISP), all interfaces
should be automatically configured as the correct type. Most implementations
already do this. 

Secondly, defaults \texttt{HelloInterval} of 10 seconds, and
\texttt{RouterDeadInterval} of 40 seconds should be used. There is also an
optional reduction of the router inactivity time from
\texttt{RouterDeadInterval} down to a minimum of \texttt{HelloInterval} + 1
seconds. 

Finally, with regard to OSPFv3 Address Families, all interfaces should use the
default value of 0 as their Interface Instance ID\@, to indicate that the
interface uses unicast IPv6.  

\section{New LSA Type}
\begin{figure}
\label{AC-LSA}
\begin{center}
	\includegraphics[width=\linewidth]{../Diagrams/Packets/ac_lsa.png}
	\caption{The new AC-LSA added by the Autoconf draft.}
\end{center}
\end{figure}

The OSPFv3 Autoconfig draft specifies a new LSA type to ensure that
there are no conflicts in Router ID across the network. This new LSA is known
as an Auto Configuration LSA and currently uses an experimental Type value. A
value will be assigned by IANA when the draft is upgraded to an RFC. 

The AC-LSA is designed to be extensible, the aim is to allow more information
to be contained in these LSAs. The structure of an AC-LSA can be seen in
Figure~\ref{AC-LSA} 

\subsection{TLVs}
\begin{figure}
\label{TLV}
\begin{center}
	\includegraphics[width=\linewidth]{../Diagrams/Packets/tlv.png}
	\caption{The Structure of a TLV contianed in an AC-LSA. Value has a variable
	size and is padded to the nearest 32 bits.}
\end{center}
\end{figure}

The payload of the AC-LSA is a set of Type-Length-Values (TLV). Each TLV
contains: 
\begin{itemize}
    \item The type, or meaning, of the data.
    \item The length of the value in bytes.
    \item The data itself.
  \end{itemize}

Diagrams of the TLV's structure can be been in Figure~\ref{TLV}

\subsubsection{Router-Hardware-Fingerprint TLV}
\begin{figure}
\label{RHWFP-TLV}
\begin{center}
	\includegraphics[width=\linewidth]{../Diagrams/Packets/rhwfp_tlv.png}
	\caption{A router-hardware-fingerprint TLV.}
\end{center}
\end{figure}
The Router-Hardware-Fingerprint (RHWFP) TLV, specified in the Autoconfig draft,
contains a unique identifier for its originating router that is valid across
the whole network. The RHWFP TLV is used to ensure that the network is free
from Router ID conflicts.

\subsubsection{Aggregated Prefix TLV}
\begin{figure}
\label{AggregatedPrefix-TLV}
\begin{center}
	\includegraphics[width=\linewidth]{../Diagrams/Packets/aggregated_prefix_tlv.png}
	\caption{An aggregated prefix TLV.}
\end{center}
\end{figure}
The Aggregated Prefix TLV is defined in the Prefix Assignment draft, it
represents a short IPv6 prefix delegation that can be used to assign prefixes
to the different subnets. 

An aggregated prefix should have a length of less than or equal to /64.
Typically an ISP will provide a /48. Aggregated prefixes have many possible
sources, the main two are DHCPv6-PD and manual configuration. 

\subsubsection{Assigned Prefix TLV}
\begin{figure}
\label{AssignedPrefix-TLV}
\begin{center}
	\includegraphics[width=\linewidth]{../Diagrams/Packets/assigned_prefix_tlv.png}
	\caption{An assigned prefix TLV.}
\end{center}
\end{figure}
The Assigned Prefix TLV is also defined in the Prefix Assignment draft. A
router adds an Assigned Prefix TLV to its AC-LSA when it makes an assignment
from an Aggregated prefix. Each assignment is made to one interface, and is a
/64 as this is the length used by SLAAC\@. Conflicts are resolved automatically
if they occur. 

\chapter{Implementation}
\section{Core Algorithms}
There are several algorithms used in Autoconfig OSPFv3 to ensure that the network
is configured correctly.

\subsection{Router-Hardware-Fingerprint Generation}
The Router-Hardware-Fingerprint is a value that is based on properties of
the router that has a high probability of being unique across the network. 

In my implementation I chose to use a concatenation of hashes of the MAC
\nomenclature{MAC Address}{Media Access Control Address} addresses of the
attached interfaces. Although, the MAC addresses are not guaranteed to be
unique as manufacturers could cut corners -- if they are not, we face much
bigger problems on the lower layers.

\subsection{Router ID Generation}
Autoconfig OSPFv3 says the Router ID should be a pseudo-random number, based on
the RHWFP\@. As the router will need to reassign its Router ID a new value if
there is a conflict, a seed value must be stored. This value is passed by
reference to the \texttt{rand\_r} function call, which modifies it to maintain
the state of the random sequence.

I found one difficult part of the project was cleanly shutting down OSPFv3,
changing the Router ID, and then bringing it back up again. 

\subsection{Router ID Conflict Resolution}
As the Router IDs are random, there is a possibility of
a collision -- two routers on the network assign themselves the same Router
ID\@. To prevent things going awry a mechanism for resolving this conflicts is needed. 

The drafts specifies two mechanisms for detecting, and resolving,
conflicts. 

\subsubsection{Local Collisions}
A local collision, one where there is an adjacency\footnote{Routers are
directly connected.} between the conflicting routers, is detected when any
valid LSA is detected to have the same Router ID but a different Link-Local
IPv6 address. To resolve this conflict, the router with the numerically lower
Link-Local Address generates a new Router ID\@. 

\subsubsection{Network-Wide Collisions}
To detect collisions that occur for routers that are not directly connected,
all apparent Self Originated AC-LSAs are inspected to ensure that the Router
Hardware Fingerprints match. If they do not then the router with the lower
Router Hardware Fingerprint must change its Router ID.

\subsection{Prefix Assignment}
When a router notices a change in the current set of AC-LSAs in the LSDB, the
prefix assignment algorithm is scheduled to run. Scheduling uses Quagga's built
in thread library and has the benefits of allowing the code to finish whatever
it was doing and the algorithm is only run once if several things schedule it
in close succession.

The algorithm examines all aggregated prefix interface pairs. If
there is not already an assignment for the pair, and the router has the
highest Router ID of all active neighbours on that link, then a
prefix is assigned to the interface from the aggregated prefix. 

There is no prescribed method in the draft for making an assignment in the
draft, beyond the fact that the prefix assigned must not be in use. I wrote my
implementation to ensure it is easy to change the method used for picking a new
prefix for the assignment. 

The method I chose is to step through the prefixes in numerical order, the
disadvantage is a fairly high likelihood of collision with a router elsewhere
in the network. This was advantageous during the testing phase as it makes
these cases more common, leading to easier testing. The other advantage is that
this algorithm is guaranteed to either find a free address, or produce an error
if the address space has been exhausted. 

\subsection{Prefix Conflict Resolution}
Prefix collisions occur when either two routers have used the same prefix for
different networks, or when two routers have assigned different prefixes to the
same network. The prefix assignment algorithm is designed so that these
collisions are resolved during assignment. Routers will only accept the prefix
assignment made by the router with the highest RID of the assigning routers. If
a router detects that one of its own assignments is no longer valid, it will
deprecate it and make a new assignment. 

\subsection{Making an Assignment}
\label{pending_list}
The prefix assignment draft specified that a hysteresis period of 20 seconds
before an assignment is committed to. The draft does not explain exactly what
is meant by this. My implementation uses a list to indicate that a prefix is
pending, so that if it needs to be deleted it can be deleted immediately. After
the 20 seconds is up, the prefix is removed from the pending list and Router
Advertisements (RA) are emitted containing the prefix.
 
\subsection{Deprecating an Assignment}
When the prefix assignment is run, all existing prefix assignments are first
marked invalid. As the algorithm runs, the prefixes that are found to be valid
are marked as such. If by the end of the algorithm a prefix is still marked
invalid, deprecation is scheduled. 

If after 240 seconds has passed, the state of the prefix has not changed, the
prefix is deleted from the interfaces list of assigned prefixes and an RA is
broadcast with the lifetime set to zero. 

\subsection{ULA Prefix Generation}
If a there are no aggregated prefixes in the routing domain after 20 seconds,
the router with the highest Router ID must generate a Unique Local Address
(ULA)\nomenclature{ULA}{Unique Local Address} prefix\footnote{In the defining
RFC these are referred to as Global IDs.} to facilitate local connectivity.
ULAs are similar in concept to IPv4 Private addresses\footnote{Although it is
not recommended to use NAT with IPv6} (e.g. 192.168.1.0). 

The ULA prefix generation algorithm is defined in RFC4193 (Section
3.2)\cite{rfc4193}. It attempts to generate a unique /48 prefix, based on a
hash of a portion of a combination of the current network time (in Network Time
Protocol (NTP) \nomenclature{NTP}{Network Time Protocol} format), and an
interface's 64-bit MAC address (EUI-64).\nomenclature{EUI-64}{Extended Unique
Identifier}

Once a ULA prefix has been generated it is advertised like any other aggregated
prefix. If a router with a higher Router ID becomes reachable, or an aggregated
prefix from another source is received, the prefix is deprecated after 120
seconds. 

\subsection{Source Based Routing}
\begin{figure}
\label{MultipleISP}
\begin{center}
	\includegraphics[width=\linewidth]{../Diagrams/Network/MultipleISP.png}
	\caption{An example of a home network with mutliple ISPs.}
\end{center}
\end{figure}
In the future it may become common for households to have multiple broadband
connections. A cable connection, with an ADSL line for backup could provide
households with a resilient connection. 

To prevent various attacks\footnote{Denial of Service
(DoS)\nomenclature{DoS}{Denial of Service} attacks using spoofed source
addresses are common.}, in accordance with BCP 38\cite{bcp38}, ISPs will not
forward traffic with a source address outside the prefix allocated to that
network.  As such, if multiple ISPs are used, network traffic must be routed to
the correct ISP. Source based routing aims to solve this problem by considering
source addresses, as well as destinations. 

I implemented a proof of concept based on my autoconfig code. My source based
routing branch is incompatible with my unaltered autoconfig implementation. My
code is based loosely on drafts that specify replacement LSAs for the standard
OSPFv3 LSA types, using Type Length Values (TLVs) similar to those in the
autoconfig draft. Due to the complexity of these drafts, I decided not to
conform to them in my proof of concept.

My implementation works by exploiting the Linux Kernel's ability to have
multiple routing tables. A routing table is created for each source address
range that will be routed. The system's policy table is then used to select
which routing table to use based on the source address.  

Quagga's Zebra daemon has the ability to manipulate any routing table. I
extended the Zebra protocol's message for adding entires to the routing table
to include the source address, and then chose the routing table based on a hash
of this address. Unfortunately Quagga does not offer the ability to manipulate
the policy table. To overcome this I used of \texttt{iproute2}'s \texttt{ip
-6 rule} command, which I called from a bash shell script, invoked using the
\texttt{system} function. 

This proof of concept demonstrates that Quagga can be extended to implement
source based routing. It is not a complete solution -- firstly it is not
interoperable with other implementations as it does not conform to the draft.
Secondly using the \texttt{system} system call is not secure, elegant or
portable. 

\section{Quagga}
Quagga is written in C89, following the GNU coding
standards\cite{gnucodestandards}. These rules help to improve the portability
of the code, meaning that if there is a C compiler, and the operating system is
Unix-like, then Quagga is likely to compile for it.  Quagga does not
support Microsoft Windows\footnote{Or WinDoze as the GNU coding standards would
put it}.

\subsection{Style}
At first I found some of the stylistic guidelines of the GNU coding standards
unusual and difficult to read, the following contributed to this:
\begin{itemize} 
  \item Space between function a call and its parameter list's opening parenthesis.
  \item Return type declaration on the line above the name of the function.  
  \item Opening curly brace of a block on a new line.
\end{itemize}

When I first opened one of Quagga's source files, it took me a short while to
realise that I was looking at a function declaration. Fortunately as time went
on, reading and writing this formatting became second nature to me. 

Being written in C89, Quagga is not object oriented. The structure of the code
however, does resemble object oriented programming. \texttt{struct}s are used
extensively to represent the different ``objects'' in the program, many
functions operate on these \texttt{struct}s, in a similar way methods that
act on the objects they belong to.

I found that working on a large and complex program was difficult at times,
determening the correct level of abstraction for each function was tricky.  I
felt I had to work around the lack of inheritance and polymorphism to write
clean code. 

When writing in a low level language like C, especially when the program will
have very long up times, ensuring that it is free from memory leaks is
important. If memory that is allocated using the \texttt{malloc} system call is
not deallocated using \texttt{free} then it will become unusable. This unusable
memory will build up, eventually the program will crash as the operating system
is unable to allocate it more memory.
 
\todo{Cite the Hacking Guide?}

\subsection{Quagga's Library}
Quagga tends to avoid relying on external libraries. There is a ``lib'' folder
in the source code repository that contains non-protocol specific code. This
library provides commonly used data structures and algorithms to the protocol
implementations. 

\subsubsection{Thread}
I made heavy use of the Thread library item. Threads are a similar to
operating system threads. They consist of pointers to functions that will be
called upon some event, or after a set time. These threads are an application
level construct. The program  executes as one process, only one thing is
being executed at any given time. Although this means the multicore
architecture of modern CPUs \nomenclature{CPU}{Central Processing Unit} is not
exploited, it simplifies development, removing the need to worry about thread
safety.

\subsubsection{LinkList}
Linklist was another useful element of the library. It implements a doubly
linked list -- a series of nodes with pointers to the next and previous node in
the list. Many of the data structures that I required in my program were
variable length lists, and Linklist worked well for this. Linklist also offers
a macro \texttt{ALL\_LIST\_ELEMENTS} that can be used like to Java's enhanced
\texttt{for} loop\footnote{Variables to hold pointers to the current and next
node make it look messy though.}. Linklist's main drawback is that due to
limitations of C, the data is referenced by a typeless pointer, and therefore
is not type safe. If the developer is careful this should not be an issue. 

\subsubsection{Prefix}
Another library component is Prefix. It represents an IPv4 or IPv6 address, and
offers an array of functions to perform on them. One of the most
useful features is the ability to convert between C strings
(\texttt{char *}) and \texttt{in6\_addrs} (the representation used in Unix-like
kernels). This is useful for user interface and debugging messages.
There are also macros available for testing whether two IP addresses were the
same -- the \texttt{memcmp} function is used by these macros. I extended Prefix
to include a function that determined whether or not a given Prefix contained
another Prefix.

\subsubsection{MD5}
The other interesting thing in the library is MD5
\nomenclature{MD5}{Message-Digest Algorithm 5}, which generates cryptographic
hashes. For the generation of ULAs, SHA-1 \nomenclature{SHA-1}{Secure Hash
Algorithm 1}, a similar but more secure hash, is recommended. As I did not have
a SHA-1 implementation, MD5 made an acceptable substitute for the time being. 

\subsection{vtysh}
\texttt{vtysh} \nomenclature{vtysh}{The Command Line Interface for Quagga} is a
Virtual Teletype shell. It is the primary interactive user interface for
Quagga. \texttt{vtysh} can be accessed locally on the router using the
\texttt{vtysh} command, or remotely using telnet. \texttt{telnet} connections
are made directly to the routing protocol daemon, i.e. \texttt{telnet ::
ospf6d}.  

The interface is based on Cisco's router's. Commands like
``\texttt{show ipv6 ospf6 database detail}'' can be used. ``\texttt{sh ipv os
da de}'' is also a valid representation of the same command as it is
unambiguous. The aforementioned command will list out, in full detail, the
contents of all of the LSAs that \texttt{ospf6d} knows about. 

I added commands to \texttt{vtysh} that allow the user to list and add
prefixes. Details of this can be found in Appendix~\ref{vtysh}. 

\section{Router Advertisements}
Router Advertisements (RA) \nomenclature{RA}{Router Advertisement} are part of
the Neighbor Discovery Protocol (NDP), \nomenclature{NDP}{Neighbor Discovery
protocol} and are used to advertise network prefixes, and the address and
routing domains of the router making the advertisement. Upon receiving
an RA, a host should generate IPv6 addresses within the prefix being advertised. 

Typically routers use \texttt{radvd} \nomenclature{radvd}{The Router
Advertisement Daemon} to generate RAs. I feel coupling Quagga to another
piece of software would be undesirable. My reason is firstly, one cannot
be sure the package will always the most popular choice. Secondly, if the
API for \texttt{radvd} changes, then Quagga would also need to be changed. And
finally there do not seem to be any clean ways of interfacing with
\texttt{radvd}: the daemon must be halted, configuration files updated and then
started up again.

\todo{Should I have a whole section on Zebra \& Zclient?} 
Fortunately Quagga has its own, slightly limited, code for emmiting RAs. This
functionality is part of Zebra, the code responsible for updating the routing
table. The \emph{correct} way of accessing this functionality from within the
routing protocol implementation (i.e. \texttt{ospf6d}), is to send a message
using the library component \texttt{zClient}, which is then received and
enacted upon by \texttt{zServ}, part of Zebra. 

\todo{Add to the appendix pictures of these messages} 
Initially the only way to set whether router advertisements were on or off, and
to set which prefixes the router advertised, was to use the \texttt{vtysh}. I
extended the \texttt{zClient-zServ} protocol to include messages that expose
this functionality to the routing protocol implementations. 

\section{Endians}
An ongoing issue in my project was the storage of integers.
\ x86 architecture, found in modern desktop processors, is little endian.
This means a value represented as multiple bytes, is stored with the least
significant (lowest in value) byte first. On the contrast
when transmitting data using the IP protocol, network byte order
is defined as big endian -- the most significant bytes are transmitted
first. 

I found had to be careful to convert to the correct endian. I spent a long time
debugging problems that arose from getting endians wrong. Fortunately for all
the standard sized data types there are functions to convert the endian.
\texttt{htons} converts host shorts (unsigned 16 bit integers) to network
shorts; \texttt{ntohs} does the opposite.  \texttt{htonl} and \texttt{ntohl}
work on longs (unsigned 32 bit integers). 

The positive of having different endians on the network and the development
architecture is that ensures endians are considered. If the development platform
used the same endians as the network then it would be very easy to forget about
endians altogether. It would then cause serious problems when the software is
ported to a platform that uses different endians to the network.
    
\chapter{Testing}
To ensure that my code worked as I intended it to, it needed to be
tested. 

\section{Assertions}
To ensure that my code failed early if something was wrong, I systematically
peppered my code with assertions. Assertions are statements that insist that
their predicate is true -- If the assertion is correct, execution continues as
normal, otherwise the program will halt.

Assertions in my code check for circumstances that should under no
circumstances occur. This ensures that the code never executes in an invalid
state, producing incorrect results. 

\section{Unit Testing}
\todo{Generate some kind of unit test report}
I created unit tests to ensure that my code behavioured as expected, my unit
tests focused heavily on the edge cases. Creating unit tests gave me confidence
that my code did what I intended it to do, however, it did not ensure that my
intentions were correct.

Although there are unit testing frameworks available for C, I decided against
using them.  I felt using external libraries  would go against the ethos of
Quagga, and my testing strategy did not fit the design of most popular
libraries. 

\todo{I think this would be a good place to crack out some diagrams}
\subsection{Test Cases}
Test cases are specified by \texttt{struct}s containing the parameters of
the system, this includes the number of interfaces and a set of LSAs that
existed in the system. 

The test cases are used to construct a mock environment that resembles
Quagga during normal execution. This is done by populating the \texttt{struct}s
Quagga uses with fake data. The prefix assignment is run for each of the test
cases. 

\subsection{Expected Values}
Each test case has a corresponding expected value, consisting of a
series of system parameters and conditions. Once a test case has been run, if
the state of the mock environment is consistent with the expected value, the
test passes, otherwise it fails. 

\section{Netkit} 
\begin{figure}
\label{NetkitTopology}
\begin{center}
	\includegraphics[width=\linewidth]{../Diagrams/Network/MainNetkit.png}
	\caption{The topology of the netkit network I used for most of my testing.}
\end{center}
\end{figure}
Netkit is a tool that allows developers to easily test network software and
configuration. Netkit uses User Mode Linux (UML) \nomenclature{UML}{User Mode
Linux} to run many Virtual Machines (VMs) \nomenclature{VM}{Virtual Machine}
on a single desktop PC. 

Netkit presents the user with an \texttt{xterm} window for each VM\@. These VMS
behave like normal linux computers. They have virtual interfaces, that act much
like real interfaces. Netkit allows these virtual interfaces to be joined
together through collision domains -- setting two interfaces to the same
collision domain is like plugging two hosts into the same network switch.
Netkit also provides a special type of collision domain called a ``tap''; this
can be used to share the host computer's network connectivity with the Netkit
network.

Netkit provided me with an indispensable testing environment. I was able to run
an instance of Quagga on each virtual machine, and control them through
\texttt{vtysh}. Having seven terminal windows on a single computer is far more
convenient than having seven separate devices that must be configured
separately. Netkit has a huge advantage over using \texttt{ssh} or
\texttt{telnet} -- it is not dependent on network connectivity, so you are free
to make changes to the network without losing your remote connection.

Compiling the code from within Netkit is also much faster than on the Alix2d3s,
the difference was between several hours on the Alix2d3s and several minutes in
Netkit. This was because rather than using a 500MHz AMD Geode\footnote{A low
power System-on-chip designed for the embedded market in around 2002}, Netkit
used my desktop PC's 3.3GHz Intel i5 2500k\footnote{A mid/high end desktop
processor from 2011} for compiling. 

I found Netkit to be a great tool for understanding how networks work.
Translating my knowledge of network theory to an ability to configure real
networks was tough. Trivial things often turn out to be the issue -- an example
is setting \path{/proc/sys/net/ipv6/conf/all/forwarding} to 1, allowing IPv6
traffic to be forwarded. It is not an important piece of theory, but if it is
not done, routing will not work. 

\todo{Stick a screen shot in the appendix :)}
\todo{Also probably want a diagram of my main test lab}
\todo{On top of that maybe add a ``Manual'' for Netkit to the Appendix} 

\section{radvdump}
\texttt{radvdump} is a tool that shows the contents of all
received router advertisements. It is designed to be used with
\texttt{radvd} and outputs the contents of a RA in the style of a
\texttt{radvd} configuration file. 

I used \texttt{radvdump} to ensure that Quagga was successfully sending RAs and
the contents were as expected. 

\todo{Add an example of one of these to the appendix}

\section{Tunnel Broker}
\todo{Add a more detailed description of how a tunnel is set up and works in the Appendix.}
Using Hurricane Electric's (HE) Tunnel broker service, I was able to give my
network global connectivity. An HE tunnel,
provides you with a single /64 IPv6 prefix. It is possible to request a larger
/48 allocation. The tunnel is configured at their end to route any traffic
that is destined to your prefix, to the endpoint that you specified.  

\todo{Maybe Appendix would like to have the route table entries to do this?}
By configuring a static route forwarding all traffic for the allocated /48
prefix into the Netkit network, then setting up Quagga within the Netkit
network, using the allocated prefix as its aggregated prefix, I was able to
make the IPv6 addresses globally reachable. Using a remote server
(\texttt{uglogin}) I was able to ping the virtual devices, demonstrating that
the devices had global addressability.

\section{Alix2d3}
Although I was confident that testing my code using Netkit was
sufficient to show it working correctly, I ran my code on two Alix2d3 devices.
This offered me more tangible proof that my code functions correctly. It also
allowed me to do things that home users are likely to do, such as plugging the
routers together in arbitrary configurations.

Testing on the Alix2d3s was much more difficult and time consuming that testing
with Netkit, so I avoided doing the bulk of the testing this way. I employed a
black box testing approach, plugging the devices together and checking that
everything seemed to be working. 

\section{Interoperability}
One of the goals of my project was to verify that the implementations produced
by other developers was compatible with the code that I produced. 

Both of the implementations I tested against were based on Bird. As Bird and
Quagga both implement the same specifications for OSPFv3, I felt it was safe to
assume that the unmodified versions Bird and Quagga would be compatible. 

\subsection{Ben's Implementation}
Ben Paterson's implementation was written in C, I found it very easy to install onto my
Netkit testbed -- it required nothing more than cloning the Git repository and
then compiling the fork of Bird on one of the netkit VMs. 

Ben's implementation correctly implements the Autoconfig draft. However, I
discovered that my implementation was wrong: The router Hardware fingerprint is
defined as 32 bytes, I had used as 32 bits. I modified my implementation to
correct this, also adding support for padded TLVs. After making the appropriate
changes to my implementation, the two worked well together for this draft.

Unfortunately, this implementation does not correctly implement the Prefix
Assignment draft. A fourth TLV type not present in the Prefix Assignment draft
is used. This TLV contains information about an interface, and is then followed
by a series of Assigned Prefix TLVs minus any interface information. 

Ben's implementation uses an interesting algorithm for assigning prefixes:
It begins by picking a prefix from the address range at random, it then checks
that this prefix has not already been used. If it conflicts then it picks
another at random. To ensure that this random process does not continue
indefinitely, after a certain number of attempts, the algorithm falls back to
picking the addresses sequentially. 

\subsection{Markus' Implementation}
Markus Stenberg's implementation takes an unusual approach to extending Bird. A
fork of Bird has been created called Bird-Ext-LSA\@\footnote{Not to be confused
with External LSAs in OSPFv3 that carry information from other protocols.},
which offers a mechanism that allows LSAs to be processed by an separate
program. 

Bird-Ext-LSA is complimented by hnet-core (Homenet Core), a project that makes
use of Bird-Ext-LSA to implement, among other things, the Autoconfig drafts\@.
hnet-core is written in the scripting language lua. Running this code was a
challenge -- I had problems creating the right environment; the code has  many
dependencies. \texttt{Luarocks} helped install some of these. I did eventually
get the code running on my Netkit testbed.

Once I had the code running, I found it was incompatible with my implementation
without alteration. To enable compatibility with a version created by the
draft's author (Jari Arrko), a different AC-LSA Type number was used -- this
confused me at first, but once spotted it was simple to fix.

The other compatibility issue was that this implementation interpreted the TLVs
length to include the TLV header.The draft explicitly states that the length is
of the value alone, so I corrected the code, to allow myself to continue
testing. The implementation relied on the interface ID being appended to the
TLV header rather than being part of the value, so the changes were harder than
they should have been.

After all this was fixed, there seemed to be a good level of compatibility,
with prefixes from one finding their way into the other.

\section{Future Testing}
To increase the confidence in my code to the level expected of a
commercial product, it would be wise to undergo further testing. 

One method of testing is to use a commercial network protocol compliance
verification suite, such as IxANVL or Spirent. These ensure that the protocol
conforms to the specification. This is currently not possible for my project as
the implemented protocol is not yet included in these suites.  Although there
is an open source library for testing OSPFv2, I was unable to find any open
source libraries for testing OSPFv3 that could be easily extended to test
autoconfiguration. 

Moving beyond this, some level of user testing should be undertaken before
publicly releasing this code, and integrating it into Quagga. Typically, code
undergoes closed Alpha testing by a small number of developers. Followed by
Beta testing, where it is released to the public, but without guarantees of
correct functionality.

\chapter{Project Management}
\section{Tools}
A personal goal for my project was to use of lots of new tools. During the
project, I ended up using a wide variety of tools. Some of the tools I had used
before, but using them on a large project meant that I made use of features that
I did not know existed before.

\subsection{Git}
I chose to use Git as my version control system. This choice was
partially because Quagga uses Git, however there are many other good reasons for
using Git, or any other Distributed Version Control System (DVCS).
\nomenclature{DVCS}{Distributed Version Control System} 

Firstly, version control is essential to a software project as it enables the
developer to keep track of changes over time and revert to old versions if
necessary.  DVCSs are superior to traditional centralised version control
system such as Subversion (SVN) because they allow work to be done without internet
connectivity and commits are much faster so tend to be made more often. 

I kept the source code for my project in a public Github repository\footnote{My
project's Github can be found at
\url{https://github.com/edderick/quagga\_zOSPF}}. This allowed me to work in
many different locations without worryin about synchronising files. Although
I made regular backups of my project, I believe that Github is an incredibly
safe place to store my code. Appendix~\ref{GithubStats} contains some
diagrams and stats generated by Github about my project. 

\subsection{Trello}
During my project, I used Trello, a web based kanban style task
management system\footnote{My Trello can be found at
\url{https://trello.com/board/50b53fc0ec17d1c75f003c17}}.
Trello provides cards than can be placed into lists and then moved around
between them. The lists can have titles such as ``Thinking'', ``Doing'' and
``Done''. Trello was very useful for tracking what tasks I had done, and what
tasks still needed doing. 

\subsection{\LaTeX}
I chose to produce all of the documentation associated with this project using
\LaTeX, because it is a powerful tool that allows me to focus on writing my
report rather than typesetting it. It also tends to be more robust for longer
documents than WYSIWYG editors such as Microsoft Word. Additionally I made use
of tools and packages to help me produce the report. (e.g. TexCount and BibTeX)

\section{Risk Assessment}
\todo{Tabulate?}
Throughout my project there were many risks, I ensured I took necessary
precautions to limit the damage caused by these risks.

When using the power supplies, there was a risk that I could get an electric
shock. I took the power supplies to be PAT tested, as the power supplies are
double insulated, a quick visual inspection by a trained member of staff
reassured me that they would be safe to use.

Around 01/12/2012, ECS had a fairly major network outage due to a hardware
failure. I was unable to do other coursework because I needed to use the
software within ECS\@. Although I was fortunate enough to avoid any network
outages, my use of DVCS meant that it would not have been an issue.

Fire or other unforeseeable disaster could have damaged to my data, in 2005 the
old Mountbatten building burned to the ground. Although this risk had a very
low likelihood, its severity was high.  To ensure that, if something like this
happened, I would not have been affected, I made frequent backups of my
work, and stored it within Github which is hosted on remote servers.

As my project involves using specialist hardware, there was a possibility that
the hardware could break. Fortunately the Alix2d3s did not cease to function
during the project. However, it turned out that they were not as important to
the project as I had initially thought since I did most of my testing and
development using Netkit.

I could have misjudged the difficulty of the project and been unable to complete
the project on schedule. Fortunately this was not an issue. The iterative
approach that I took to the project enabled me to implement a proof of
concept of an extension.  

\section{Gantt Charts}
As the project progressed my schedule changed slightly, the Gantt charts
reflect these changes.

Firstly, I did not account for the high work load over the Christmas holiday. I
expected to work on my third year project but instead spent time on other
coursework and preparation for January exams. This meant that implementation
began later than planned, by working hard on my project I was able to complete
it on time; then go on to do an extension. 

At the start of the project I though I would use Alix2d3s for testing, so
did not allocated time for setting up Netkit. Although this did take time
initially, it saved me time in the long run by reducing the time taken for
testing. 

I decided that user testing was not be appropriate for this project. The aims
of the project were more about the feasibility and correctness of the drafts
that their usability. Users should never need to know that this software is
running in their homes, so do not need to evaluate it. 

\pagebreak
\begin{landscape} 
\subsection{Interim Gantt Chart}
This Gantt chart shows how I had spent my time, and estimated that I would
spend the remainder of my time when I submitted my Progress Report in December. 

\begin{center}
  \hspace*{-0.75cm}
  \includegraphics[width=1.\linewidth]{../Gantt/EvenBetterDec.png}
\end{center}

\pagebreak

\subsection{Final Gantt Chart}
This Gantt chart shows how my project actually progressed.

\begin{center}
  \hspace*{-0.75cm}
  \includegraphics[width=1.\linewidth]{../Gantt/EvenBetterApril.png}
\end{center}

\end{landscape}

\pagebreak

\chapter{Evaluation}

\section{What Went Well?}
\todo{Mention Requirements or Goals}
In general the project was a success. Although it was difficult, I managed to
complete my implementation on schedule. I feel the software I have produced is
functional, robust and maintainable.  Hopefully after code review it will be
accepted as part of Quagga.

I enjoyed the opportunity to work in a language and style of programming with
which I had little experience.  Because of this I have learnt lots about how
procedural programming differs from an object oriented approach. Over the
project my C programming ability improved drastically. I am much more confident
in my understanding of pointers and \texttt{struct}s, and in turn, the low
level operation computers.  

I am now more confident working on large projects. Before this project, I
looked at open source code bases and felt intimidated by their size and style.
I was unsure of where the code for these applications even began executing.
Having completed this project, I realise that the statements that make up large
projects, are no different from ones I am familiar with from my own work --
there is rarely anything that when taken in isolation, I am unable to
understand.

I learnt a lot about the software projects that exist to support networking.
Not only did I discover Quagga, but I also learnt about many other projects. I
used competing routing daemon implementations: Bird and XORP\@. I also read
about Cisco's proprietary implementations of these protocols. 

I used many other pieces of network software: I made heavy use of
\texttt{iproute2}, the package that has superseded \texttt{ifconfig} as the de
facto way to control TCP/IP networking in Linux. I also experimented with
\texttt{radvd} the Router Advertisement Daemon, and played with WIDE-DHCPv6. 

\section{What Could Have Gone Better?}
Although the project generally went well, perfection is rarely obtainable, and
there are things that if I did this project again, I would do differently. 

Firstly, due a high workload from other modules at the end of Semester One, I
did not properly dig into the project until later than I would have liked -- in
hindsight I should have chosen  less coursework intensive optional modules.
Another factor was my lack of confidence about jumping in at the deep end.
Working on this project I have learnt that one of the best ways of understanding
a code base is to try adding a feature to it.

I also feel that I could have become more involved in the discussion about the
drafts, again I think I was held back by a lack of experience and confidence,
having never been involved with the Internet Engineering Task Force or any
other international standards body before. 
 
\chapter{Conclusions and Future Work}

\section{Conclusions}
In this project I have shown that it is possible to extend Quagga to implement
zero configuration OSPFv3; Along the way I learned a lot about networks. 

\subsection{Ambiguities in the Drafts}
There are a few areas of the drafts that I feel are slightly ambiguous or under
prescriptive. I plan on sending the following suggestions to the authors of the
drafts.

The following comments refer to the Autoconfiguration draft
\cite{draft-ietf-ospf-ospfv3-autoconfig-02}:
\begin{enumerate}
	\item In section 5.1 the draft says: 
		%\begin{quote}

		``An OSPFv3 router implementing this specification should assure that the
		inadvertent connection of multiple router interfaces to the same physical
		link in not misconstrued as detection of a different OSPFv3 router with a
		duplicate Router-ID.'' 
		%\end{quote}

		 I feel the draft should be more prescriptive on this issue.  My
		 implementation checks the router's interfaces link local addresses,
		 comparing them to the source address of the packet, I am not sure this is
		 correct.
	\item Section 7 states:
		%\begin{quote}

		``It is RECOMMENDED that OSPFv3 routers supporting this specification
		also allow explicit configuration of OSPFv3 parameters as specified
		in Appendix C of [OSPFV3].''
		%\end{quote}

		It does not mention what to do upon a conflict between an automatically
		generated and a manually configured Router-ID: The two routers decide
		independently whether they will change. If the autoconfigured router
		determines that the other must change its RID, we will be left in a
		conflicting state.
	\item  The non-neighbour case (5.2) for RID collisions covers all collisions,
		it should cover the neighbour case (5.1). A justification for the existance
		of the neighbour case might be beneficial.
\end{enumerate}

These suggestions are about the prefix assignment draft
\cite{draft-arkko-homenet-prefix-assignment-03}:
\begin{enumerate}
	\item Section 8 says:
		%\begin{quote}

		``A new prefix assignment within an aggregated prefix SHOULD NOT be
		committed before the router has waited NEW\_PREFIX\_ASSIGNMENT seconds for
		another prefix or reachable OSPFv3 router to appear.''
		%\end{quote}

		The draft doesn't state exactly what is meant by commited. As mentioned in
		Section \ref{pending_list}, I used a pending list to ensure a young
		conflicting assignment could be deleted quickly.
	\item The draft does not indicate how assignments should be chosen. Using a
		random or sequential value seems obvious, but it may be possible to reduce
		routing table entries by employing a scheme that facilitates route
		aggregation. 
\end{enumerate}

\section{Continuing the project}
Now the project is over, I plan to continue the work I have been doing.  My
goal is to have the code that I have written merged back into the Quagga
mainline. Before this can happen, the code will need to go through a cleanup
stage and a series of code reviews to ensure that it is of sufficient quality,
and is a good fit with the Quagga project.

I also hope to be able to attend the 87th IETF meeting during the summer
holidays. I have applied for the ECS Student Development fund, to help me fund
the trip. If I am able to go it will give me the opportunity to discuss future
work, and participate in interoperability tests.

\section{Future Work}
My project had a large scope, and as it is in a very current research area,
based on highly extensible drafts, there is a large amount of work that could
be based on what I have produced. 

This project is a good candidate for extension by a Google Summer of Code
student, or a future Third Year Project. 

\subsection{DHCPv6-PD}
Ideally an implementation of OSPFv3 Prefix Assignment should be able to handle
aggregated prefixes delegated by DHCPv6-PD\@. I decided early on in the project
that this was beyond the scope of the project. 

There are two ways I would suggest implementing this functionality. The first
is to make use of an existing DHCPv6 Client such as WIDE-DHCPv6, and hook into
its PD message functionality. The other would be to add functionality to Quagga
itself to handle the DHCPv6-PD messages. 

I feel implementing this as part of Quagga is the better solution as it does
not couple Quagga to other projects. However, it would be likely to require
more work to achieve.

\subsection{Source Based Routing} 
My proof of concept code (See Section \ref{SourceBasedRouting}) shows that it
is possible to do source based routing by extending Quagga. A more formal
implementation of source based routing would be beneficial to the community.
Ideally it would conform to the drafts, and be consistent with Quagga's
existing coding style. 

\subsection{Multi-hop Service discovery}
Service discovery is very popular in modern networks. Major use cases include
finding media streamers and printers on a home network. At present, since the
service discovery messages are not forwarded by the router, to discover a
service, a host must be on the same subnet.

It would be possible to extend Quagga to flood service discovery
messages across the whole network, enabling services on one subnet, to be
discoverable from another. Uses for this include: monitoring home appliances from
within the home workstation network, and displaying images on a public screen
from a guest network.

There are currently no drafts indicating how to do this. Ideally Apple Inc.\@
would extend their service discovery platform, Bonjour, to use this kind of
technology.

\subsection{Domain Name Services}
In a typical single subnet IPv4 home network, DNS servers  are learned through
DHCP from an ISP, or manually configured. In a multi-subnet home network,
mechanisms are needed for propagating the DNS severs available to the network.
The prefix assignment draft recommends supporting stateless DHCPv6 and Router
Advertisement otions.

Stateless DHCPv6 works by running a DHCPv6 server that maintains no state about
its clients. It simply serves configuration details such as the DNS server
addresses. The draft recommends that each router acts as a DHCPv6 relay so
hosts deep within the network can obtain DNS information.  Enabling DHCPv6
should be simple as packages (Such as WIDE-DHCPv6, Dibbler) already exist. 

Router advertisements contain two types of listing for discovering DNS servers.
First, listings for Recursive DNS Servers (RDNSS)
\nomenclature{RNDSS}{Recursive DNS Server}: Servers that will recursively
resolve Domain Names. Secondly, DNS Search Lists (DNSSL)
\nomenclature{DNSSL}{DNS Search List}, a list of servers that can be used to
resolve local names. Adding this functionality to my implementation would be
more difficult that adding stateless DHCPv6 as Quaggas Router advertisement
code would need to be modified to deal with these options.

The draft does not mention whether configuration learned through either method
should be consolidated, and advertised through both. If this is required,
complexity would increase.

\subsection{Security Considerations} 
Built in security was a feature of OSPFv2 but was removed in OSPFv3 in favour
IPSec (IP Security). IPSec is difficult to deploy so is not suitable for home
users. RFC6506 reintroduces OSPF security through the use of an Authentication
Trailer\cite{rfc6506}. Implementing this RFC would add Security to
Autoconfigured networks. Users would need to choose a secret key and enable
authentication trailers to prevent unauthorised routers joining their home
network.  

\section{Deployment in the Home}
As IPv6 becomes more common place, and IPv4 is slowly phased out, this project can 
actually be used in the home. For home users to want to switch to IPv6 there must 
be features they cannot get from IPv4 -- I feel that this project 
implements one of those features.

It does not however stop at home users, ISPs also need to be on board.
Currently in the UK there is only a handful of ISPs providing IPv6 addresses to
their customers. SixXS list just 7 UK ISPs, the most popular of these being
Andrews \& Arnold.

These problems are outside of my control, hopefully with time, the
large ISPs will overcome the inertia of switching to IPv6.
 
\raggedbottom
\pagebreak

\printnomenclature
\raggedbottom


\addcontentsline{toc}{chapter}{Nomenclature}
\pagebreak

\cleardoublepage
\phantomsection
\addcontentsline{toc}{chapter}{Bibliography}
\bibliography{FinalReport}{}
\bibliographystyle{ieeetr}

\appendix 
%\appendixpage
\chapter{OSI Model}
\label{osi}
  \begin{center}
  \begin{tabular}{|lc|}
  \hline
  7: &
  Application Layer \\
  & (e.g.\ DNS, DHCP) \\
  \hline
  6: &
  Presentation Layer \\
  & (e.g.\ telnet) \\ 
  \hline
  5: &
  Session Layer \\
  & (e.g.\ RPC) \\
  \hline
  4: &
  Transport Layer \\
  & (e.g.\ TCP, UDP) \\
  \hline
  3: &
  Network Layer \\
  & (e.g.\ IPv4, IPv6) \\
  \hline  
  2: &
  Data Link Layer \\
  & (e.g.\ MAC) \\ 
  \hline  
  1: &
  Physical Layer \\
  & (e.g.\ Ethernet, Wi-Fi) \\
  \hline
\end{tabular}
\end{center}

The seven layer OSI model is a logical grouping of the types of protocols found
in computer networks. The protocols in each layer tend to dependend only on the
protocols in the layer layer directly below them. 

The TCP/IP model is a similar, but simpler model of networking.

\chapter{Additional zClient-zServ Messages}
\label{zClient}
I extended the zClient-zServ protocol to expose some of the features,
previously only available through \texttt{vtysh}, to the routing protocol
implementations. 

\begin{center}
	\includegraphics[width=0.9\linewidth]{../Diagrams/Packets/nd_no_suppress.png}
\end{center}

The first of these messages is one to enable sending Router Advertisements. It
works by stopping NDP messages from being suppressed. There is also a
corresponding message to turn these RAs off again. 

\begin{center}
	\includegraphics[width=0.9\linewidth]{../Diagrams/Packets/ipv6_nd_prefix.png}
\end{center}

The next message adds a network prefix to the RAs. This is used to advertise a
new assigned prefix. There is also a similar message to remove the prefix.


\begin{center}
	\includegraphics[width=0.9\linewidth]{../Diagrams/Packets/addr_add.png}
\end{center}

The final message adds an IPv6 Address to a specified interface. In my
implementation this was used to associate a prefix with an interface for the
routing table calculations. Again, there is also a message to remove the
address.

\chapter{Netkit}
\begin{center}
	\includegraphics[width=\linewidth]{../Diagrams/Netkit/NetkitScreenshot.png}
\end{center}

Above is a screen shot of a typical Netkit session.  

Netkit instances are started with \texttt{vstart \$NAME}.

A lab is a collection of VMs and collision domains, described in a
configuration file. Labs are started with \texttt{lstart}. It is often useful
to allocate the VMs in the lab more memory using the flag \texttt{-o-M}n, where
n is the size of the memory to allocate in megabytes.

\chapter{radvdump}
\label{radvdump}
This \texttt{radvdump} output shows the contents of the router advertisements
made by one of my quagga instances. The prefix fc35:6225:4b8:6::/64 is a ULA
that was generated automatically by my implementation.

\lstinputlisting{../Issues/radvdump.md}

\chapter{GitHub Stats}
\label{GithubStats}
GitHub offers various graphics that can be used to visualise the progress and
patterned in the project. 

\section*{GitHub Commit Graph}
\begin{center}
	\includegraphics[width=\linewidth]{../Diagrams/Stats/GitHubCommitGraph.png}
\end{center}

This commit graph shows all of the commits that I made to my fork of Quagga.
The graph shows that I began working on my implementation in early February and
worked steadily, peaking at the beginning of the Easter vacation. The second
peak is when I began working on testing.

\section*{GitHub Punchcard}
\begin{center}
	\includegraphics[width=\linewidth]{../Diagrams/Stats/GitHubPunchCard.png}
\end{center}

This punch card shows the time of day, and day of the week that I made 
my contributions. The graph is for my documentation repository as the code
repository contained many commits by other authors.

\chapter{Alix2d3}
As the project progressed it became clear that they Alix2d3s were not as useful
as was originaly believed, thanks to discovery of the briliant software package
Netkit. I did however undertake testing on the Alix2d3s. If someone wished to
repeat my experimentation, the following guides would be helpful. 

\section*{Ubuntu Installation Guide}

 This guide\cite{germanGuide} offered a comprehensive list of steps that need to
 be taken to install Ubuntu Linux on a PC Engines alix2d3\cite{alix2d3}.  The article is
 written in German but using Google translate I was able to follow it as it is
 mainly a list of shell commands.

 The process begins with installing the compact flash card as a device on an
 existing Ubuntu desktop PC\@. Next the card is partitioned using \texttt{fdisk},
 the card is then formatted as ext2 (a Linux file system type) using
 \texttt{mk2fs}. The card is then mounted in the file system, using
 \texttt{mount} and a small installation of Linux is copied to the card using
 \texttt{debootstrap} to provide a base system.

 Several settings and devices are linked to the flash cards mount point and
 \texttt{chroot} is used to simulate booting into the new install. The required
 configuration files are edited (e.g.\@ network interface settings) and essential
 software packages (such as \texttt{Vim}, \texttt{SSH}, \texttt{sudo}, and
 \texttt{APT}) are installed. A boot loader such as \texttt{GRUB} is also
 installed and configured.

 Next, the flash card is safely removed, and  inserted into the alix2d3. The
 alix2d3 is then powered on. Using a USB to Null Model cable connecting a PC to
 the alix2d3's serial port, a terminal connection can be established using {\bf
 cu}. I had a few problems using this device (\texttt{ttyUSB0}) but I was able to
 fix this problem using \texttt{chown}. This terminal connection can be used to
 aid the boot process and ensure that there are no Magic ELF errors. Once the
 alix2d3 is up and running it is possible to disconnect the serial cable and
 access it over \texttt{SSH} alone.

 \section*{Quagga Installation Guide}

 \subsection*{Normal Installation}
This procedure turned out to be far more difficult that anticipated. Firstly,
all the dependencies must be installed. The easiest way I have found to do this
by using:

\texttt{sudo apt-get build-dep quagga}

This should fetch all the packages that Quagga depends on. These do not need to
be built from source, since I have not modified them.\ \texttt{libtool} and
\texttt{libreadline} need to be installed separately. 

Next it should be possible to run \texttt{\@./bootstrap}, this creates the
configure script; This can be run with the command:

\begin{quote}
\texttt{sudo \@./configure \\ --sysconfdir=/usr/local/quagga \\ --localstatedir=/usr/local/quagga}
\end{quote}

The configure script will prompt you to create a user called quagga:

\begin{quote}
\texttt{sudo adduser quagga} \\
\texttt{sudo mkdir /usr/local/quagga} \\
\texttt{sudo chown quagga:quagga /usr/local/quagga} 
\end{quote}

It is now time to run \texttt{make}. See the next section for more information
about compiling for the Alix2d3. I issues with making zebra which I fixed
by manually editing the Makefile in \texttt{\@./zebra}, adding \texttt{-lcap}
to \texttt{LIBS}\@. This issue no longer occurs with the latest release of
Quagga.

Running \texttt{sudo make install} should install everything -- the binaries for the daemons
should be located in \texttt{/usr/local/sbin}.

Finally a little bit of configuration is required. \path{/usr/local/quagga}
contains the configuration files, either make a real config file for the
daemons (zebra and ospf6d), or just create file containing ``password test''.

You should now be able to run \texttt{ospf6d}. If you can't, try \texttt{ldconfig}.

\subsection*{Cross Compilation}
\label{cross_compile}
Since Alix2d3s are very slow, building Quagga can be incredibly slow. The make
parts alone can take around 15 minutes. (XORP had an even slower compile time,
in the order of hours) 

To overcome this issue it is possible to compile the source code on a much
faster machine (e.g.\@ desktop pc) and copy it across. 

Since modern machines use a 64bit architecture rather than 32bit like the
Alix2d3s, the compiler must be set up correctly. The cflag \texttt{-m32
-march=i386} can be used. This can be done by appending
\texttt{--with-clfags=`-m32 -march=i386i'} to \texttt{\@./configure}.

I initially had problems with this compilation. I resolved the issues by going
back to basics and compiling a simple ``helloworld'' program for the
Alix.  Turns out I was missing \texttt{gcc-multilib}. This can be installed
with \texttt{apt}.

\chapter{vtysh}
\label{vtysh}
Commands are added using the macro \texttt{DEFUN} and the function \texttt{install}.

I added the commands:
\begin{itemize}
	\item \texttt{show ipv6 ospf6 prefix aggregated} \\
		Used to display all aggregated prefixes in the system.
	\item \texttt{show ipv6 ospf6 prefix allocated} \\ 
		Used to display the only aggregated prefixes that were allocated to this
		router only.
	\item \texttt{show ipv6 ospf6 prefix assigned X} \\
		Used to display the prefixes that have been assigned to a given interface
		by the prefix assignment algorithm. 


	\item \texttt{enable $\rightarrow$ configure terminal 
		$\rightarrow$ ipv6 allocate-prefix X:X::X:X/M} \\
		Used to manually allocate a new aggregated prefix to the routers.
	\item \texttt{enable $\rightarrow$ configure terminal
		$\rightarrow$ no ipv6 allocate-prefix X:X::X:X/M} \\
		Used to remove a previously allocated aggregated prefix.
\end{itemize}
\pagebreak

\begin{landscape}
\chapter{Structure of ospf6d}
\label{ospf6d}
\begin{center}
  \includegraphics[width=0.9\linewidth]{../Diagrams/UML/quaggaOSPF6D/ospf6d_simple.png}
\end{center}

\end{landscape}

\chapter{Test Results}
\label{testResults}
Of the twenty test cases that I produced, all of them now pass. The following is an
example of the output of my unit testing code: 

\begin{lstlisting}
%\ldots%
Testcase 5  %{\color{green} OK}%
Toal interface count: 2 
 Aggregated Prefix: fc00::/48
 Aggregated Prefix: fc00:1::/48
Total Aggregated Prefix count: 2 
 eth0's Assigned Prefix count: 2
  Assigned Prefix: fc00:0:0:1::/64
   Not Valid
   Deprecation Thread
  Assigned Prefix: fc00:0:0:3::/64
 eth1's Assigned Prefix count: 2
  Assigned Prefix: fc00:0:0:2::/64
  Assigned Prefix: fc00:1::/64
   Pending Thread
Total Assigned Prefix count: 4 
%\ldots%
Total Failed Tests: 0
\end{lstlisting}

\chapter{File Listings}
As this project is based on an existing software project, it would be
unreasonable to list all of the files contained in my submitted archive.
Instead, the following is a list of the most notable files: Those that I have
created, modified or made heavy use of throughout the project. The best way to
see what I changes I have made is to view my project on
GitHub.\footnote{\url{https://github.com/edderick/quagga_zOSPF}}

\section*{Netkit Labs}
The folder \path{netkit} contains a series of Netkit labs. I created these lab
folders to contain the settings of Netkit Labs.

In each lab folder, \path{lab.conf} contains the definition of the collision domains.
When a VM boots, the file consisting of its name suffixed with \path{.startup}
is run, and the files within the folder structure rooted at the folder with its
name are copied to its root directory.

\section*{Quagga}
The majority of the work that I have done is contained within
\path{quagga_zOSPF/ospf6d/}. I made modifications to many of the file in this
folder. \path{ospf6_auto.h} and \path{ospf6_auto.c} are my own additions. I
also made substantial additions to \path{ospf6_intra.h} and
\path{ospf6_intra.c}.

I also made alterations to several files in \path{quagga_zOSPF/lib/}. Most
notably: \path{prefix.h}, \path{prefix.c}, \path{zclient.h} and
\path{zclient.c}. In \path{quagga_zOSPF/zebra/}, I modified \path{zserv.c}.

My unit testing code can be found in \path{quagga_zOSPF/tests/}, contained
entirely within the file \path{ospf6d_autoconf_test.c}. When compiled, this
generates the file \path{testospf6dautoconf}.

\section*{hnet-core.diff}
The file \path{hnet-core.diff} contains the changes that I made to hnet-core,
Markus' autoconf OSPFv3 implementation, to increase its compatibility with my
own.

\chapter{Original Project Brief}

\section*{A Tool to Simplify Network Administration in the Modern Home}

\subsection*{Problem}

As computing becomes more and more ubiquitous, home networks are becoming
increasingly complex. In the future we may see households containing a very
large number of devices, performing functions we can't yet even imagine. 

As home networks grow in both number of hosts and bandwidth requirements, the
task of administering such networks will inevitably become more complicated. 

Unfortunately, most homeowners don't have degrees in computer science, so will
find it hard to set up effective networks in their homes. 

\subsection*{Goals}

The project's main aim is to simplify the administration of complex home
networks. This goal shall be realised by providing users with a system that
either reduces the complexity of managing a complex, multi-subnet, home
network, or providing tools for monitoring the activity of the network. 

\subsection*{Scope}


The project will involve either:
\begin{itemize}
\item Implementing a zeroconf OSPF routing system based on an open source
  routing implementation such as Quagga or XORP.

\item Create a traffic monitoring system that can report on bandwidth usage,
  network flows and perform some form of malware detection. With the
  possibility of implementing a front-end to allow easy QoS configuration. 

\end{itemize}

Both of these projects will involve the use of lightweight Linux boxes to
perform their tasks. The project that will be pursued will be decided upon by
week four. 




\end{document}
