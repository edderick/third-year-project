% !TEX TS-program = pdflatex
% !TEX encoding = UTF-8 Unicode

% This is a simple template for a LaTeX document using the "article" class.
% See "book", "report", "letter" for other types of document.

\documentclass[11pt]{article} % use larger type; default would be 10pt


\usepackage[utf8]{inputenc} % set input encoding (not needed with XeLaTeX)

%%% Examples of Article customizations
% These packages are optional, depending whether you want the features they provide.
% See the LaTeX Companion or other references for full information.

%%% PAGE DIMENSIONS
\usepackage{geometry} % to change the page dimensions
\geometry{a4paper} % or letterpaper (US) or a5paper or....
% \geometry{margin=2in} % for example, change the margins to 2 inches all round
% \geometry{landscape} % set up the page for landscape
%   read geometry.pdf for detailed page layout information

\usepackage{graphicx} % support the \includegraphics command and options

% \usepackage[parfill]{parskip} % Activate to begin paragraphs with an empty line rather than an indent


%%% PACKAGES
\usepackage{booktabs} % for much better looking tables
\usepackage{array} % for better arrays (eg matrices) in maths
\usepackage{paralist} % very flexible & customisable lists (eg. enumerate/itemize, etc.)
\usepackage{verbatim} % adds environment for commenting out blocks of text & for better verbatim
\usepackage{subfig} % make it possible to include more than one captioned figure/table in a single float
% These packages are all incorporated in the memoir class to one degree or another...

%%% HEADERS & FOOTERS
\usepackage{fancyhdr} % This should be set AFTER setting up the page geometry
\pagestyle{fancy} % options: empty , plain , fancy
\renewcommand{\headrulewidth}{0pt} % customise the layout...
\lhead{}\chead{}\rhead{}
\lfoot{}\cfoot{\thepage}\rfoot{}

%%% SECTION TITLE APPEARANCE
\usepackage{sectsty}
\allsectionsfont{\sffamily\mdseries\upshape} % (See the fntguide.pdf for font help)
% (This matches ConTeXt defaults)

%%% ToC (table of contents) APPEARANCE
\usepackage[nottoc,notlof,notlot]{tocbibind} % Put the bibliography in the ToC
\usepackage[titles,subfigure]{tocloft} % Alter the style of the Table of Contents
\renewcommand{\cftsecfont}{\rmfamily\mdseries\upshape}
\renewcommand{\cftsecpagefont}{\rmfamily\mdseries\upshape} % No bold!

%%% END Article customizations

%%% The "real" document content comes below...

\title{COMP3020 - Individual Project \\ Project Proposal
}
\author{Student - Edward Seabrook \\ Supervisor - Tim Chown}
\date{} 


\begin{document}
\maketitle

\section*{A Tool to Simplify Network Administration in the Modern Home}

\subsection*{Problem}

As computing becomes more and more ubiquitous, home networks are becoming increasingly complex. In the future we may see households containing a very large number of devices, performing functions we can't yet even imagine. 

As home networks grow in both number of hosts and bandwidth requirements, the task of administering such networks will inevitably become more complicated. 

Unfortunately, most homeowners don't have degrees in computer science, so will find it hard to set up effective networks in their homes. 

\subsection*{Goals}

The projects main aim is to simplify the administration of complex home networks. This goal shall be realised by providing users with a system that either reduces the complexity of managing a complex, multi-subnet, home network, or providing tools for monitoring the activity of the network. 

\subsection*{Scope}


The project will involve either:
\begin{itemize}
\item Implementing a zeroconf OSPF routing system based on an open source routing implementation such as Quagga or XORP.

\item Create a traffic monitoring system that can report on bandwidth usage, network flows and perform some form of malware detection. With the possibility of implementing a front-end to allow easy QoS configuration. 

\end{itemize
}

Both of these projects will involve the use of lightweight Linux boxes to perform their tasks. The project that will be pursued will be decided upon by week four. 

\end{document}

